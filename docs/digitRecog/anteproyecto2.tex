\documentclass[a4paper, 11pt, oneside]{article}


% idioma
\usepackage[utf8]{inputenc}
\usepackage[spanish]{babel}

%tablas
\usepackage{booktabs}

%rotar tablas
\usepackage{rotating}

%color tablas
\usepackage{colortbl}



%espaciado
\usepackage{setspace}
\onehalfspacing
\setlength{\parindent}{0pt}
\setlength{\parskip}{2.0ex plus0.5ex minus0.2ex}


%margenes según n. icontec
\usepackage{vmargin}
\setmarginsrb           { 4.0cm}  % left margin
                        { 3.0cm}  % top margcm
                        { 2.0cm}  % right margcm
                        { 2.0cm}  % bottom margcm
                        {   0pt}  % head height
                        {0.0 cm}  % head sep
                        {   9pt}  % foot height
                        { 1.0cm}  % foot sep

% inserción url's notas de pie.
\usepackage{url}


% Paquetes de la AMS:
\usepackage{amsmath, amsthm, amsfonts}
\addto\captionsspanish{\def\refname{\textsc{Bibliografía}}}

\newcommand\portada{
	\begin{titlepage}
		\begin{center}
			{\large \bf DESARROLLO DE UN SISTEMA DE RECONOCIMIENTO ÓPTICO DE CARACTERES PARA CELULARES, QUE FUNCIONE BAJO CONDICIONES CONTROLADAS }
			\vfill
% 			{\large\bf PRESENTADO POR \par}
			{\large\bf SEBASTIÁN GÓMEZ GONZÁLEZ \par}
			{\large\bf SANTIAGO GUTIERREZ ALZATE \par}
			\vfill
			{\large\bf UNIVERSIDAD TECNOLÓGICA DE PEREIRA  \par}
			{\large\bf FACULTAD DE INGENIERÍAS \par}
			{\large\bf INGENIERÍA DE SISTEMAS Y COMPUTACIÓN \par}
			{\large\bf PEREIRA\par}
			{\large\bf SEPTIEMBRE DE 2010 \par}
		\end{center}
	\end{titlepage}
}


\begin{document}
\portada
%\contraportada

	\clearpage
	\section{Título}
	Desarrollo de un sistema de reconocimiento óptico de caracteres para celulares, que funcione bajo condiciones controladas.
	
	\section{Formulación del problema}
	Los problemas visuales afectan a una gran parte de la población, tanto en Colombia\footnote{Según el DANE en su gran censo nacional realizado en el año 2005, se estima que el 6.3\% de la población colombiana sufre de alguna discapacidad. De estas personas con discapacidad se estima que el 43.4\% tienen dificultades para ver aun con lentes.} como a nivel mundial\footnote{La organización mundial de la salud estima que en el mundo hay 161 millones de personas con problemas visuales, de los cuales 37 millones son ciegos.}, estos traen consigo consecuencias devastadoras para las personas que los sufren, afectando su calidad de vida significativamente. Entre todos los problemas que sufren las personas con discapacidades visuales, uno de los más críticos es el del acceso a la información; esto debido a que la mayoría de la información los seres humanos reciben del entorno llega a través de los ojos. Los problemas para obtener información causan que las personas con discapacidades visuales se queden rezagadas con respecto a los demás miembros de la sociedad, impidiendo así que tengan las mismas oportunidades de vida que una persona con sus cinco sentidos intactos. Generalmente, esto causa un efecto de exclusión y de segregación en estas personas debido a que son vistas como una carga, tanto por si mismas como por las personas que los rodean. 
	
	Entre las soluciones que existen en el mercado para este problema, no se encontró un software económico para que las personas invidentes puedan tener acceso rápido a documentos escritos. Esto es debido a que el software tradicionalmente construido para reconocer caracteres ópticamente no esta diseñado para ser usado en dispositivos móviles. Después de una búsqueda en más de 20 artículos científicos relacionados al tema de reconocimiento óptico de caracteres, y de revisar la documentación técnica de varios OCRs libres, no se encontró un sistema de reconocimiento de caracteres por un medio óptico que haya sido desarrollado para teléfonos inteligentes y que sea de código libre y abierto.

	Existen grandes diferencias entre un computador de escritorio y un teléfono inteligente, entre ellas se pueden resaltar las siguientes:
	\begin{itemize}
	\item La memoria y la capacidad de procesamiento son muy reducidas en los teléfonos inteligentes con respecto a los computadores de escritorio.

	\item En los computadores de escritorio las imágenes se obtienen por medio de un escáner, logrando condiciones de iluminación muy buenas y uniformes. En los teléfonos inteligentes las imágenes se obtienen por medio de una cámara fotográfica, haciendo más relevantes diversas variables, como por ejemplo: la inclinación del texto, iluminación no uniforme, deformaciones elásticas del texto debido a la forma del papel en reposo.

	\item El tamaño de la memoria cache es mucho más pequeño en los teléfonos inteligentes. Como resultado de esto, los algoritmos diseñados para computadores de escritorio pueden ser muy lentos en los teléfonos inteligentes, pues se hicieron pensando en caches de varios Mbytes.

	\item Un computador de escritorio normalmente tiene instaladas múltiples librerías de propósito general que son utilizadas por distintos programas, un teléfono inteligente, en cambio, solo provee las librerías más básicas.
	\end{itemize}

	Sin embargo una ventaja de los teléfonos inteligentes, es que son por naturaleza dispositivos
	de alta conectividad. Cada vez los planes de telefonía son mas económicos y hay mas disponibilidad
	de redes inalámbricas en diferentes sitios públicos como institutos educativos y bibliotecas. 
	
	\subsection{Definición del problema}
	No existe \footnote{Zhou, Gilani and Winkler. Open Source OCR Framework Using Mobile Devices}
    un sistema de reconocimiento óptico de caracteres que halla sido desarrollado para 
    celulares inteligentes, que sea de código abierto, y con un porcentaje de reconocimiento 
    satisfactorio sobre imágenes tomadas con cámara.
	
	\clearpage

	\section{Justificación}
	A partir de la incidencia del problema, una forma de ayudar a que las personas con discapacidades visuales tengan una mejor calidad de vida es lograr que puedan tener acceso a la información que se encuentra en documentos escritos, tales como cartas, libros, y volantes. Un sistema que solucionase este problema, le posibilitaría a las personas con discapacidades visuales mejorar enormemente su calidad de vida, ya que les permitiría una mejor educación, al poder acceder a los mismos libros de texto que el resto de la sociedad, y no solo, como actualmente sucede, a los pocos libros que se encuentran disponibles en formato braille.

	Para dar una solución efectiva al problema se requiere de algún sistema que le permita a los discapacitados visuales acceder a la información escrita de una manera rápida y fácil, lo cual implica gran movilidad. Si bien existen sistemas de reconocimiento óptico de caracteres para computadores de escritorio, estos no poseen dicha movilidad, porque además del computador, también necesitan de un escáner para funcionar. Así pues, para que un sistema de este tipo pueda ser utilizado en múltiples lugares, se requiere de algún dispositivo que sea: pequeño, fácilmente transportable y lo suficientemente poderoso para hacer el trabajo en un tiempo razonable. Los celulares comunes cumplen las primeras dos características, pero no la tercera, ya que su poder de procesamiento es varias veces inferior al de un computador común. Sin embargo, gracias a la conectividad que poseen los celulares, estos pueden enviar la imagen por la red para que sea procesada por un servidor que tenga la capacidad de computo necesaria.	
	Cabe anotar que la capacidad de proceso de los celulares también a incrementado en los últimos años, lo que permitiría hacer parte del procesamiento en el celular, con el objetivo de reducir el ancho de banda requerido para enviar la imagen al servidor y ahorrar dinero a los usuarios.

	Además de esto, un sistema móvil para reconocer texto de documentos impresos presentaría los siguientes beneficios respecto de los sistemas tradicionales como las impresoras braille y los lectores basados en escáner:
	\begin{itemize}
	\item Un menor costo, al no ser necesario adquirir un computador de escritorio, escáner o impresora braille.

	\item Portabilidad: ninguno de los sistemas anteriormente mencionados esta diseñado para ser llevado en todo momento por el usuario, dificultando el acceso al conocimiento e información que no se encuentre en el lugar físico en el que se encuentra alguno de estos sistemas.
	
	\item Una solución económica que pueda ejecutarse en un dispositivo móvil, sin necesidad de hardware adicional, haría posible que cada invidente pudiera tener su propio sistema de OCR personal, mientras que los otros sistemas usualmente (por su costo) son adquiridos por instituciones en cantidad limitada.

	\item Aportar al conocimiento: Al desarrollar este aplicativo se tendrán que hacer pruebas experimentales cuyos resultados aporten a otras investigaciones.
	\end{itemize}
	\clearpage

	\section{Objetivo}
	\subsection{Objetivo general}
	Desarrollar un sistema prototipo de reconocimiento óptico de dígitos numéricos bajo condiciones especificadas, 
	usando para el reconocimiento una distribución normal multivariable con el teorema de Bayes
	y un vector de características invariantes (Momentos de Hu y Fluzzer).
	\subsection{Objetivos específicos}
	\begin{itemize}
	\item Realizar un estudio del funcionamiento de la distribución normal multivariable en el problema de reconocimiento usando
	el teorema de Bayes.
	\item Realizar la extracción de características de un conjunto de imágenes de dígitos. Las características a extraer serán
	invariantes a la rotación, translación y escalamiento (Momentos de Hu y Fluzzer).
	\item Diseñar e implementar una aplicación que integre estos algoritmos y permita el reconocimiento óptico de dígitos.
	\item Realizar pruebas de la aplicación implementada.
	\item Realizar un análisis comparativo de los resultados obtenidos.
	\end{itemize}
	\clearpage
	
	\section{Marco referencial}
	
	\subsection{Marco conceptual}
	\begin{itemize}
    \item Análisis de documentos: Proceso que consiste en encontrar características de un documento,
    y dividirlo en sus componentes básicos ( párrafos, líneas, palabras y caracteres).\footnote{
	BREUEL Thomas M. The OCRopus Open Source OCR System, Proceedings IS\&T/SPIE 20th Annual Symposium, 2008.}
    \item Reconocimiento óptico de caracteres (OCR): Un OCR como proceso de cómputo, es un conjunto de
    algoritmos cuyo resultado esperado es la extracción de información  que este contenida dentro de
    una imagen, que sean parte de un sistema de escritura humano, y que pueda ser codificada
    posteriormente como texto entendible.\footnote{Ibid., p.2}
    \item Manhatan layout: Es una forma de distribución de los elementos de un documento escrito o
    impreso, en los que los elementos como párrafos o imágenes se pueden inscribir
    dentro de rectángulos de área mínima, de modo que la pendiente entre dos lados sea igual para
    todos los rectángulos.\footnote{Ibid., p.5}
    \item Escala de grises: Una imagen en la que cada píxel es representado como un valor entre 0 y
    255, siendo 0 negro y 255 blanco y cualquier otro valor intermedio algún tipo de gris. \footnote{
    BREUEL Thomas M. Efficient implementation of local adaptative thersholding techniques using integral 
    images. Image Understanding and Pattern Recognition (IUPR) Research Group, 2008.}
    \item Binarización:	Es el proceso de asignar un valor binario (uno o cero por ejemplo) a una
    variable dependiendo de	ciertas condiciones, en el contexto del procesamiento de imágenes se le
    llama binarización a el proceso de cambiar el valor de un píxel dentro de un formato de imagen
    estándar a 1 o 0, dependiendo generalmente de un valor límite generalmente llamado threshold.
    \footnote{Ibid., p.2}
    \item Filtrado de imagen: Consiste en remover de una imagen elementos indeseables, que no
    son interesantes para el procesamiento requerido de la imagen y en algunos casos dificultan
    o imposibilitan obtener los resultados deseados.
    \footnote{Ibid., p.2}
    \item Píxel adyacente: Dos píxeles $(x_1,y_1)$, $(x_2,y_2)$ son adyacentes, si $|x_1-x_2| \le 1$
    y $|y_1-y_2| \le 1$
    \item Componente conectado: Después de tener una imagen binarizada (blanco y negro), un
    componente conectado es un conjunto de píxeles negros adyacentes rodeados por un fondo blanco.
    \item Características o momentos invariantes: Son vectores numéricos que se calculan a partir
    de una imagen, y dependen de la forma de la imagen. Pero son independientes de la rotación,
    translación y escalamiento.
    \footnote{Shafait, Keysers y Breuel. Performance Evaluation and Benchmarking of Six-Page segmentation
    Algorithms, IEEE Transactions on Pattern Analysis And Machine Intelligence, 2008, p.6}
   
	\end{itemize}
	
	\subsection {Antecedentes}
	La tecnología actual ha permitido a las personas invidentes tener acceso a información
	y a conocimiento al que antes no tenían acceso; Entre estas tecnologías se encuentran los
	lectores de pantalla, impresoras y sistemas refrescables de Braille, entre otros.
	Un proyecto interesante desarrollado en la Universidad Tecnológica de Pereira, permite
	a los invidentes reconocer imágenes con colores a través de vibraciones a diferentes
	frecuencias (Llamado Proyecto IRIS); Para ello, usan un guante con imanes 
	permanentes y una matriz de electroimanes conectada al computador por el puerto USB.
	\newline \newline
	En el área de acceso al texto, existen sistemas que pasan el texto en ASCII a voz
	conocidos como sistemas \textit{Text-To-Speech}. Y también sistemas para el computador
	que permiten pasar texto en imagen a texto en ASCII, conocidos como OCR; Estos funcionan
	usualmente tomando una imagen de un escáner. Según pruebas de eficacia realizadas por
	la UNLV, los OCRs libres con mejores resultados son	\textit{Tesseract} y el 
	\textit{Ocropus}. Cuando el problema cambia de tomar la imagen con un escáner a tomarla
	con una cámara, se deben seguir diferentes enfoques, en 
	\footnote{Ray Smith. Progress in camera-based document image analysis.} se halló que los
	retos más importantes son la proyección de la imagen en perspectiva en el plano y la 
	resolución requerida para poder reconocer bien los caracteres.
	En cuanto a hacer que el OCR sea realizado por un celular, se encontraron investigaciones
	como \footnote{Zhou, Gilani and Winkler. Open Source OCR Framework Using Mobile Devices}, 
	en la que se trabajaban con imágenes de muy baja resolución (640x480) y como resultado solo
    pueden reconocer textos muy cortos, tales como avisos
	y carteles,	pero no documentos completos.
	En \footnote{Senda, Nishiyama, Asahi and Yamada. Camera-Typing Interface for Ubiquitous
	Information Services}, para tratar de equilibrar el hecho de tener una cámara de baja
	resolución, toman varias imágenes de baja resolución en lugar de una sola, uniéndolas
	luego para lograr una mejor resolución. Esto sin embargo conlleva a un incremento la
	carga de procesamiento, haciendo que no sea posible el procesamiento dentro del celular.
	Así que restringieron su solución para que usando Internet se enviaran las imágenes a un
	servidor, que fuera este quien las procesara y enviara los resultados.\newline
	Cabe anotar además que ninguna de las tecnologías anteriormente mencionadas para hacer
	el OCR en el celular, fue diseñada para ser usada por personas invidentes. Ya que el 
	primero no fue diseñado para reconocer documentos, y el segundo requeriría que la persona
	invidente supiera con anterioridad como está la estructura del documento para poder mover
	la cámara del celular en el orden en el que está el texto en el documento.
	
	\subsubsection{Proyectos de grado relacionados en la Universidad Tecnológica de Pereira}
	Si bien en Ingeniería en la Universidad Tecnológica de Pereira no se ha planteado un proyecto 
    de grado directamente sobre reconocimiento óptico de caracteres, estudiantes de la maestría en
    Ingeniería Eléctrica construyeron un sistema para el reconocimiento de algunos caracteres 
    y dígitos manuscritos. Se intentaron dos maquinas de aprendizaje diferentes:
	el perceptrón multicapa (comúnmente conocido como red neuronal) y la maquina de vectores de
	soporte. Sus resultados demostraron que las maquinas de vectores fueron superiores a las redes
    neuronales en todas las pruebas, así que estas maquinas de aprendizaje podrían ser de gran 
    utilidad en la construcción de un sistema de reconocimiento óptico de caracteres.\newline
    
	En temáticas relacionadas, se encontraron los siguientes proyectos de grado desarrollados por 
    estudiantes de Ingeniería Eléctrica:    
    En la tesis \footnote{Jaime Varela Rincón, Johan Eric Loaiza Pulgarín. Reconocimiento de palabras
    aisladas mediante redes neuronales sobre FPGA} se utilizaron redes neuronales para reconocimiento
    de voz en hardware sobre una FPGA. 
    Si bien la implementación fue en hardware y no en software, sus resultados permiten inferir que es 
    posible implementar maquinas de aprendizaje sofisticadas (como las redes neuronales) con un bajo uso
    de memoria y con relativa eficiencia, lo cual da esperanzas para una posible implementación de las 
    misma en un dispositivo móvil. También es de anotar que las redes neuronales demostraron porcentajes 
    de aciertos significativamente mayores que los obtenidos por los clasificadores bayesianos lineales.
	\newline \newline
	En el tema de procesamiento de imágenes, se encontró un proyecto de grado para contar transeúntes en
    una imagen utilizando un sistema de visión artificial ``Registro de transeúntes en tiempo real utilizando
    un sistema de visión artificial sobre un ambiente controlado''. Considerando el bajo nivel de
	resolución (imágenes de 324 x 240 píxeles a 30 frames por segundo) y el hecho de que la cámara no
	fue calibrada, se identificaron correctamente muchos transeúntes, de lo que se puede inferir que los 
    sistemas de visión artificial funcionan correctamente con cámaras económicas. Los resultados muestran que
	el uso de histogramas puede ser de gran utilidad a la hora de distinguir componentes conectados de 
    píxeles. Así mismo, también permiten entender distintos problemas que se pueden presentar a la hora de
    separar estos componentes correctamente, como por ejemplo la identificación de varios transeúntes como 
    uno solo. Si bien en el reconocimiento óptico de caracteres se distinguen letras y no personas, problemas 
    similares a los evidenciados en este proyecto se presentan al separar los caracteres en un sistema de 
	OCR.\newline
	
	Finalmente, en el tema de procesamiento de señales, en ``Acondicionamiento de Señales Bioeléctricas''
	se utilizan filtros para adquirir señales bioeléctricas utilizando filtros análogos y digitales.
	En este proyecto se lograron filtrar señales muy pequeñas y finas, como las producidas en el cuerpo humano, 
	mediante filtros digitales, y se obtuvieron errores pequeños.
	Aunque las señales bioeléctricas son muy diferentes a las imágenes, es posible 
    convertir estas ultimas en señales y, por tanto, los filtros digitales expuestos en este proyecto podrían 
    ser de utilidad a la hora de eliminar errores en imágenes tomadas por una cámara, ya que logran filtrar
	correctamente señales con gran cantidad de ruido.\newline
	
	\subsection {Marco teórico}
	\subsubsection{Reconocimiento de caracteres y análisis de documentos}
	En esta sección se hablará de algunas investigaciones realizadas en el área
	de reconocimiento de caracteres y de análisis de documentos.
	Las etapas necesarias para llevar a cabo el proceso de análisis de documentos
	y reconocimiento de caracteres son: Binarización de la imagen, análisis y
	segmentación del documento y reconocimiento óptico de caracteres (OCR).
	
	\subsubsection{Binarización de imagen}
	Esto se refiere a tomar una imagen en escala de grises y convertirla a una 
	imagen en blanco y negro. También se puede ver como separar la parte que nos interesa
	(foreground o primer plano) de la parte que no nos interesa (background o fondo).
	Una técnica que da buen resultado es la binarización de Sauvola,
	que realiza una binarización basada en un análisis local de la imagen. El hecho de
	que sea local le brinda una mejor binarización en condiciones de luz variable. Los 
    resultados obtenidos en el artículo "Efficient Implementation of Local Adaptative 
    Thersholding Techniques Using Integral Images" indican que esta técnica puede ser
    optimizada para lograr un tiempo de ejecución comparable con las técnicas globales de 
    binarización (al correr en O(N) \footnote{La función $O(f)$ se usa para indicar que la complejidad
    algorítmica es una función $f(N)$ que depende de la talla del problema $N$. Sea $T(N)$ el tiempo de 
    ejecución del algoritmo para una talla de problema $N$, entonces existe una constante $K$ tal que 
    $\forall N ( K*f(N) < T(N) ) $}, siendo N el numero de píxeles de la imagen),
    conservando	las ventajas de ser local, por lo que seria un buen candidato para implementarlo en un 
    proyecto para desarrollar un sistema de reconocimiento óptico de caracteres.
	
	\subsubsection{Análisis del documento}
	Según \footnote{Song Mao, Azriel Rosenfeld and Tapas Kanungo. "Document Structure Analysis
	Algorithms: A literature Survey"}, esta etapa puede ser vista como un análisis sintáctico,
	en el que el documento se divide en sus partes estructurales como un árbol de análisis
	sintáctico. Como todo análisis gramatical, se puede seguir los enfoques
	\textit{buttom-up} y \textit{top-down}. Un enfoque \textit{buttom-up} comenzaría desde
	los píxeles y luego los relacionaría en grupos mas grandes (Caracteres, Palabras, Lineas
	o Párrafos). Un enfoque \textit{top-down} por el contrario, comienza con la imagen de un
	documento y comenzaría a dividirla hasta llegar a sus componentes.
	Cada algoritmo tiene limitaciones, ventajas y distintos requerimientos de máquina.
	Una investigación realizada sobre diferentes algoritmos \footnote{Shafait, Keysers and Breuel.
	"Performance Evaluation and Benchmarking of Six-Page segmentation Algorithms"}, indica que los
	mejores resultados sobre diferentes documentos, con distintas estructuras físicas,
	fueron para Docstrum y para Voronoi, con porcentajes de error del 4.3\% y 4.7\% respectivamente.
	Otro método	llamativo es el \textit{RLSA}, que llama la atención por su sencillez y puede brindar
	buenos resultados en documentos \textit{Manhattan-Layout}.
	Probablemente la mejor solución para un sistema OCR pueda resultar de la mezcla de
	varios de los algoritmos existentes, en \footnote{Kise, Sato and Matsumoto. "Document Image 
	Segmentation as Selection of Voronoi Edges"} se observa como se pueden mejorar
	los resultados mezclando los beneficios de diferentes algoritmos para soluciones
	particulares, y cada uno de los artículos mencionados aporta elementos importantes para
    escoger cual de estos algoritmos es ventajoso para una solución particular.
	
	\subsubsection{Reconocimiento óptico de caracteres OCR}
	Esta etapa se puede clasificar como un subproblema del reconocimiento de patrones.
	Existen varias aproximaciones para solucionar el problema de reconocimiento de patrones; 
	modelos inspirados en la biología como los que usan redes neuronales. Otros
	basados en modelos matemáticos como los momentos invariables de hu, o los
	descriptores de Fourier. El principio fundamental de esta clase de algoritmos son
	los descriptores invariantes, estos son: funciones para evaluar imágenes, que retornan valores 


	parecidos o idénticos cuando se aplican	transformaciones afines a estas, como la rotación,
	el cambio de escala, la traslación, entre otras. \newline	
	
	Generalmente lo primero que se hace es tomar muchas imágenes de un solo tipo: sea una letra, un 
	número o una forma. Medir sus características aplicando descriptores invariantes, y manualmente 
    etiquetar esta imagen, (una a, un 1, un rombo etc) guardando esta información en una base de
    datos. Luego cuando el sistema se pone en marcha se obtienen las características de la imagen,
    y se aplica un algoritmo de aprendizaje sobre dichas características. Uno de los métodos mas
	sencillos consiste en tomar la distancia euclidiana normalizada entre valores de la misma
	clase y en clases diferentes. Con esto se pueden crear regiones en el hiper-espacio (un
	espacio de varias dimensiones) para las diferentes clases, y así cuando llegue una imagen
	nueva a clasificar solo se deben tomar las características y ubicar en que que región del
	hiper-espacio se encuentra este objeto. En los artículos se puede observar como  
	cada aproximación tiene sus ventajas, dependiendo de las características de las imágenes a
	procesar.
	\newpage
		
	\section{Diseño metodológico preliminar}
	
	\subsection{Hipótesis}
	Es posible desarrollar un sistema de reconocimiento óptico de dígitos usando el teorema
	de Bayes con una distribución normal multivariable y tomando como características los
	momentos invariantes de Hu y de Fluzzer al procesar una imagen de un documento que 
	cumpla con las restricciones establecidas (Ver población), tomando la imagen con
	una cámara de un celular inteligente y ejecutando el programa en un sistema de cómputo.
		
	\subsection{Tipo de investigación}
	En esta investigación se utilizará un enfoque cuantitativo.
	
	\subsection{Población}
	Los documentos impresos que solo contengan dígitos numéricos, que cumplan con las siguientes restricciones:
	\begin{itemize}
	\item Forma rectangular.
	\item Fondo blanco.
	\item Letra de color negro y con un único tipo de letra o fuente.
	\item El tamaño de los dígitos varía entre 12 y 20 puntos
	\item La inclinación de los dígitos varía entre -30° y 30°.
	\end{itemize}
	
	\subsection{Unidad de análisis}
	Fotos de textos con las condiciones mencionadas en la población, con una única fuente.

	\subsection{Muestra}
	Se tomará una muestra de 10 páginas, una página por cada dígito. Cada página contendrá entre 100
	y 120 instancias de cada dígito con una unica fuente pero con diferentes tamaños de letra en el
	rango especificado. De cada página se tomarán 5 fotos con diferentes inclinaciones en el rango
	de inclinación especificado. Las fotos deben ser tomadas desde un punto perpendicular al papel.
	
	\subsection{Variables}
	Las variables a tener en cuenta para ejecución de la aplicación son:
	\begin{itemize}
	\item Porcentaje de caracteres reconocidos exitosamente en total.
	\item Porcentaje de caracteres reconocidos exitosamente por cada dígito.
	\item Porcentaje de seguridad con el que se reconocieron los dígitos en total.
	\item Porcentaje de seguridad con el que se reconoció cada dígito.
	\item Latencia (Tiempo que se demora el OCR en comenzar a entregar salida)
	\end{itemize}
		
	\subsection{Esquema temático}
	\begin{itemize}
		\item Algoritmos de reconocimiento de caracteres y aprendizaje de máquinas
		\item Selección de un conjunto de características (Momento invariante)
		\item Comprensión del algoritmo de aprendizaje, el teorema de Bayes y distribución multivariable
		\item Diseño e implementación del reconocimiento óptico de dígitos
		\item Pruebas y análisis de resultados.
	\end{itemize}
	\clearpage
	
	\section{Personas que participan en el proyecto}
	\begin{itemize}
   \item Santiago Gutierrez\\
Estudiante Ingeniería de Sistemas y Computación\\
Universidad Tecnológica de Pereira\\
Santigutierrez1@hotmail.com
   \item Sebastián Gómez\\
Estudiante Ingeniería de Sistemas y Computación\\
Universidad Tecnológica de Pereira\\
sebasutp@gmail.com
   \item Saulo de Jesús Torres\\
Doctorado en Informática: Ingeniería de Software (Universidad Pontificia de Salamanca)\\
Universidad Tecnológica de Pereira\\
saulo.torres@utp.edu.co
	\end{itemize}
	\clearpage
	
	\section{Recursos disponibles}
	\begin{itemize}
		\item Computador portátil Dell Inspiron 1420\\
			Intel Centrino Duo, 2GB de memoria RAM
		\item Celular Nexus One, con sistema operativo Android\\
			Procesador ARM, 512 MB de memoria RAM, Cámara 5 MPixeles
		\item Base de datos de artículos publicados en IEEE explore
		\item Código fuente del OCR experimental de licencia libre Ocropus
	    \item Recursos bibliográficos de la biblioteca de la Universidad Tecnológica de Pereira y propios en el área de inteligencia artificial, reconocimiento óptico de caracteres y procesamiento digital de imágenes.
	\end{itemize}
	
	\section{Posibilidades de publicación}
	\begin{itemize}
		\item Un artículo en el reconocimiento de figuras geométricas básicas usando métodos de aprendizaje de máquinas
		\item Un artículo en la aplicación de momentos invariantes para el reconocimiento de dígitos numéricos
		\item Un artículo en el reconocimiento de dígitos usando la distribución normal multivariable
	\end{itemize}
	
	\clearpage
	
	\section{Cronograma}
	La investigación y desarrollo del proyecto tendrán una duración aproximada de 16 semanas,
	trabajando 24 horas semanales cada uno de los miembros del grupo. Para un total de 1152
	horas hombre.
	\begin{itemize}
	\item Realizar un estudio acerca de los algoritmos existentes ( 288 horas )
	\item Realizar una comparación técnica para seleccionar los mejores algoritmos ( 72 horas )
	\item Implementar cada uno de los algoritmos escogidos ( 360 horas )
	\item Diseñar una aplicación que integre estos algoritmos ( 72 horas )
	\item Realizar pruebas de la aplicación implementada ( 72 horas )
	\item Comparar el resultado obtenido por la aplicación implementada con el resultado obtenido por aplicaciones para el reconocimiento óptico de caracteres, tantos libres, como comerciales. ( 72 horas )
	\item Redacción del informe de proyecto de grado ( 144 horas )
	\item Holgura ( 72 horas )
	\end{itemize}
	\clearpage
	
\bibliographystyle{plain}
\bibliography{refs}
\nocite{*}
\end{document}
