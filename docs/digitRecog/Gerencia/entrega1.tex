\documentclass[a4paper, 12pt, oneside]{article}


% idioma
\usepackage[utf8]{inputenc}
\usepackage[spanish]{babel}

%tablas
\usepackage{booktabs}

%rotar tablas
\usepackage{rotating}

%color tablas
\usepackage{colortbl}



%espaciado
\usepackage{setspace}
\onehalfspacing
\setlength{\parindent}{0pt}
\setlength{\parskip}{2.0ex plus0.5ex minus0.2ex}


%margenes según n. icontec
\usepackage{vmargin}
\setmarginsrb           { 4.0cm}  % left margin
                        { 3.0cm}  % top margcm
                        { 2.0cm}  % right margcm
                        { 2.0cm}  % bottom margcm
                        {   0pt}  % head height
                        {0.0 cm}  % head sep
                        {   9pt}  % foot height
                        { 1.0cm}  % foot sep

% inserción url's notas de pie.
\usepackage{url}

% Paquetes de la AMS:
\usepackage{amsmath, amsthm, amsfonts}
\addto\captionsspanish{\def\refname{\textsc{Bibliografía}}}

\newcommand\portada{
	\begin{titlepage}
		\begin{center}
			{\large \bf PRIMERA ENTREGA DE GERENCIA DE PROYECTOS }
			\vfill
% 			{\large\bf PRESENTADO POR \par}
			{\large\bf SEBASTIÁN GÓMEZ GONZÁLEZ \par}
			{\large\bf SANTIAGO GUTIERREZ ALZATE \par}
			\vfill
			{\large\bf UNIVERSIDAD TECNOLÓGICA DE PEREIRA  \par}
			{\large\bf FACULTAD DE INGENIERÍAS \par}
			{\large\bf INGENIERÍA DE SISTEMAS Y COMPUTACIÓN \par}
			{\large\bf PEREIRA\par}
			{\large\bf SEPTIEMBRE DE 2010 \par}
		\end{center}
	\end{titlepage}
}


\begin{document}
\portada

	%\clearpage
	
	\begin{center}
	\section{Formulación del problema}
	\end{center}

	Los problemas visuales afectan a una gran parte de la población, tanto en Colombia\footnote{Según el DANE en su gran censo nacional realizado en el año 2005, se estima que el 6.3\% de la población colombiana sufre de alguna discapacidad. De estas personas con discapacidad se estima que el 43.4\% tienen dificultades para ver aun con lentes.} como a nivel mundial\footnote{La organización mundial de la salud estima que en el mundo hay 161 millones de personas con problemas visuales, de los cuales 37 millones son ciegos.}, estos traen consigo consecuencias devastadoras para las personas que los sufren, afectando su calidad de vida significativamente. Entre todos los problemas que sufren las personas con discapacidades visuales, uno de los más críticos es el del acceso a la información; esto debido a que la mayoría de la información los seres humanos reciben del entorno llega a través de los ojos. Los problemas para obtener información causan que las personas con discapacidades visuales se queden rezagadas con respecto a los demás miembros de la sociedad, impidiendo así que tengan las mismas oportunidades de vida que una persona con sus cinco sentidos intactos. Generalmente, esto causa un efecto de exclusión y de segregación en estas personas debido a que son vistas como una carga, tanto por si mismas como por las personas que los rodean. 

	A partir de la incidencia del problema, una forma de ayudar a que las personas con discapacidades visuales tengan una mejor calidad de vida es lograr que puedan tener acceso a la información que se encuentra en documentos escritos, tales como cartas, libros, y volantes. Esto puede ser logrado a través un sistema de reconocimiento óptico de caracteres que reconozca el texto escrito en estos documentos y le lea a la persona invidente lo que dicen. Sin embargo, para que sea una solución efectiva al problema, se requiere de un sistema que le permita a los discapacitados visuales acceder a la información escrita de una manera rápida y fácil, lo cual implica gran movilidad. Si bien existen múltiples sistemas de reconocimiento óptico de caracteres para computadores de escritorio, estos no poseen dicha movilidad, porque además del computador, también necesitan de un escáner para funcionar. Así pues, para que un sistema de este tipo pueda ser utilizado en múltiples lugares, se requiere de algún dispositivo que sea: pequeño, fácilmente transportable y capaz de hacer el trabajo en un tiempo razonable. Los teléfonos inteligentes parecen ser una buena alternativa, no obstante, existen grandes diferencias entre un computador de escritorio y un teléfono inteligente, entre ellas se pueden resaltar las siguientes:
	
	\begin{itemize}
	
	\item La memoria y la capacidad de procesamiento son muy reducidas en los teléfonos inteligentes con respecto a los computadores de escritorio.

	\item El tamaño de la memoria cache es mucho más pequeño en los teléfonos inteligentes. Como resultado de esto, los algoritmos diseñados para computadores de escritorio pueden ser muy lentos en los teléfonos inteligentes, pues se hicieron pensando en caches de varios megabytes.

	\item Un computador de escritorio normalmente tiene instaladas múltiples librerías de propósito general que son utilizadas por distintos programas, un teléfono inteligente, en cambio, solo provee las librerías más básicas.

	\end{itemize}

	La diferencia mas importante, sin embargo, es que en los computadores de escritorio las imágenes son obtenidas por medio de un escáner, logrando condiciones de iluminación muy buenas y uniformes, mientras en los teléfonos inteligentes las imágenes se obtienen por medio de una cámara fotográfica. Esta diferencia es muy importante para el reconocimiento óptico de caracteres, en especial si quien toma la foto es una persona invidente, ya que en este caso diversas variables como la inclinación del texto, la iluminación no uniforme, y las deformaciones elásticas del texto debido a la forma del papel en reposo, cobran más relevancia, pues la persona invidente podría tomar la fotografía inclinada, en el sentido contrario, o en un lugar con poca luminosidad.

	Para dar solución al problema de limitación en la capacidad de procesamiento de los teléfonos inteligentes se puede utilizar una de sus grandes ventajas: son por naturaleza dispositivos de alta conectividad. Cada vez los planes de telefonía son mas económicos y hay mas disponibilidad de redes inalámbricas, a las cuales los teléfonos inteligentes tienen acceso, en diferentes sitios públicos como institutos educativos y bibliotecas. Esta alta conectividad posibilita la construcción de un sistema de reconocimiento óptico de caracteres con una arquitectura cliente-servidor, en la cual el teléfono inteligente pueda enviar la imagen por internet y esta sea procesada por un servidor con una capacidad de computo mucho mayor. Este sistema tendrá las mismas ventajas de movilidad y rapidez deseadas. La capacidad de procesamiento de los celulares inteligentes también ha incrementado en los últimos años, lo que permitiría hacer parte del procesamiento en el celular, con el objetivo de reducir el ancho de banda requerido para enviar la imagen al servidor y por tanto el costo.

	Para solucionar los problemas asociados con la inclinación del texto que se puede producir en las imágenes tomadas por la persona invidente, se hace necesario un sistema de reconocimiento óptico de caracteres que pueda reconocer exitosamente fotos tomadas con distintos grados de inclinación o incluso en el sentido contrario.

Después de una búsqueda en más de 20 artículos científicos relacionados al tema de reconocimiento óptico de caracteres, y de revisar la documentación técnica de varios OCRs libres, no se encontró un sistema de reconocimiento de caracteres por un medio óptico que haya sido desarrollado para ser usado por personas invidentes en teléfonos inteligentes, que sea de código libre y abierto, y que solucione los problemas de inclinación y luminosidad antes presentados. 	
	
	\clearpage

	\begin{center}
	\section{Justificación}
	\end{center}
	
	Además de esto, un sistema móvil para reconocer texto de documentos impresos presentaría los siguientes beneficios respecto de los sistemas tradicionales como las impresoras braille y los lectores basados en escáner:

	\begin{itemize} 

	\item Un menor costo, al no ser necesario adquirir un computador de escritorio, escáner o impresora braille.

	\item Portabilidad: ninguno de los sistemas tradicionales esta diseñado para ser llevado en todo momento por el usuario, dificultando el acceso al conocimiento e información que no se encuentre en el lugar físico en el que se encuentra alguno de estos sistemas.
	
	\item Una solución económica que pueda ejecutarse en un dispositivo móvil, sin necesidad de hardware adicional, haría posible que cada invidente pudiera tener su propio sistema de OCR personal, mientras que los otros sistemas usualmente (por su costo) son adquiridos por instituciones en cantidad limitada.

	\item Aportar al conocimiento: Al desarrollar este aplicativo se tendrán que hacer pruebas experimentales cuyos resultados aporten a otras investigaciones.
	\end{itemize}
	\clearpage

	\begin{center}
	\section{Metas y Objetivos}
	\end{center}
	
	\subsection{Objetivo general}
	Desarrollar un sistema prototipo de reconocimiento óptico de dígitos numéricos bajo condiciones especificadas, 
	usando para el reconocimiento una distribución normal multivariable con el teorema de Bayes
	y un vector de características invariantes (Momentos de Hu y Fluzzer).
	
	\subsection{Objetivos específicos}
	\begin{itemize}
	\item Realizar un estudio del funcionamiento de la distribución normal multivariable en el problema de reconocimiento usando
	el teorema de Bayes.
	\item Realizar la extracción de características de un conjunto de imágenes de dígitos. Las características a extraer serán
	invariantes a la rotación, translación y escalamiento (Momentos de Hu y Fluzzer).
	\item Diseñar e implementar una aplicación que integre estos algoritmos y permita el reconocimiento óptico de dígitos.
	\item Realizar pruebas de la aplicación implementada.
	\item Realizar un análisis comparativo de los resultados obtenidos.
	\end{itemize}
	
	\subsection{Metas}
	
	\clearpage
	\begin{center}
	\section{Indicadores}
	\end{center}
	
	\subsection{Indicadores de monitoreo}
	
	\subsection{Indicadores de impacto}
	
	\subsection{Indicadores de resultado}
	
	\clearpage
	\begin{center}
	\section{Relación con planes y programas}
	\end{center}
	
	\subsection{Relación con el plan de la universidad}
	Bla bla bla
	
	\subsection {Rel. Colombia}
	
	
	\subsection{Rel. Rda}
	
	\newpage
		
	\begin{center}
	\section{Cosideraciones adicionales}
	\end{center}
		
	\clearpage
	
	\begin{center}
	\section{Bibliografía}
	\end{center}
	AGRAWAL, Mudit y DOERMANN, David. Voronoi++: A Dynamic Page Segmentation approach based on Voronoi and Docstrum features. INTERNATIONAL CONFERENCE ON DOCUMENT ANALYSIS AND RECOGNITION. (10: 26-29, julio, 2009: Barcelona, Spain). Memorias, 2009. p. 1011-1015
	
	O'GORMAN Lawrence. The Document Spectrum for Page Layout Analysis. \underline{En}: IEEE Transactions on Pattern Analysis And Machine Intelligence. Noviembre, 1993. vol. 15, no. 11, p. 1162-1173
	
	BREUEL, Thomas. The OCRopus Open Source OCR System. DOCUMENT RECOGNITION AND RETRIVAL. (15: 29-31, febrero, 2008: San Jose, Estados Unidos). Memorias, 2008, p. 68-150
	
	SHAFAITA Faisal; KEYSERSA, Daniel y BREUEL, Thomas. Efficient Implementation of Local Adaptative Thersholding Techniques Using Integral Images. DOCUMENT RECOGNITION AND RETRIVAL. (15: 29-31, febrero, 2008: San Jose, Estados Unidos). Memorias, 2008, p. 61-67
	
	MAO Song; AZRIEL, Rosenfelda y KANUNGOB, Tapas. Document structure analysis algorithms: A literature survey. DOCUMENT RECOGNITION AND RETRIVAL. (10: enero, 2003: San Jose, Estados Unidos). Memorias, 2003, p. 197-207
	
	SHAFAITA Faisal; KEYSERSA, Daniel y BREUEL, Thomas. Performance Evaluation and Benchmarking of Six-Page segmentation Algorithms. \underline{En}: IEEE Transactions on Pattern Analysis And Machine Intelligence. Junio, 2008. vol. 30, no. 6, p. 941-954
	
	KISE, Koichi; SATO Akinori y MATSUMOTO, Keinosuke. Document Image Segmentation as Selection of Voronoi Edges. IEEE COMPUTER SOCIETY: CONFERENCE ON COMPUTER VISION AND PATTERN RECOGNITION (66: 17-19, junio, 1997: San Juan, Puerto Rico). Memorias, 1997, p. 32-39
	
	WONG, Kwan; CASEY, Richard y WAHL, Friedrich. Document Analysis System. \underline{En}: IBM Journal of Research and Development. Noviembre, 1982. vol. 26, no. 6, p. 647-656
	
	Ray Smith. An Overview of the Tesseract OCR Engine. IEEE COMPUTER SOCIETY: INTERNATIONAL CONFERENCE ON DOCUMENT ANALYSIS AND RECOGNITION (9: 23-26, septiembre, 2007: Curitiba, Paraná, Brazil). Memorias, 2007, p. 629-633
	
	ZHOU, Steven; SYED, Gilani y WINKLER, Stefan. Open Source OCR Framework Using Mobile Devices. MULTIMEDIA ON MOBILE DEVICES (1: 28-29 enero, 2008: San Jose, California, Estados Unidos). Memorias, 2008, vol. 6821, p. 682104.1-682104.6
	
	SENDA, Shuji, et al. Camera-Typing Interface for Ubiquitous Information Services. IEEE ANNUAL CONFERENCE ON PERVASIVE COMPUTING AND COMMUNICATIONS (2: 14-17, marzo, 2004: Orlando, Florida, Estados Unidos). Memorias, 2004, p. 366-372
	
	SMITH, Ray. Progress in Camera-Based Document Image Analysis. INTERNATIONAL CONFERENCE ON DOCUMENT ANALYSIS AND RECOGNITION (7: 3-6, agosto, 2003: Edimburgo, Escocia). Memorias, 2003, p. 606-617
	
	HU, Ming-Kuei. Visual pattern recognition by moment invariants. \underline{En}: IRE Transactions on Information Theory. Febrero, 1962, vol. 8, no. 2, p. 179-187
	
	SEUL, Machael , O'GORMAN Lawrence y SAMMON, Michael J. Practical Algorithms for Image Analysis. Nueva York: Cambridge University Press, 2000. 301 p.
	
	FLUSSER Jan, SUK, Tomas Suk y ZITOVA, Barbara. Moments and Moment Invariants in Pattern Recognition. Sussex, Reino Unido: Jhon Wiley \& Sons Ltd, 2009. 291 p.
	
	ALPAYDIN, Ethem. Introduction to Machine Learning. Estados Unidos: MIT Press, 2004. 415 p.
	
	CHERIET, Mohamed; KHARMA, Nawwaf y LIU Cheng-Lin. Character recognition systems: Guide for students and practitioners. New Jersey, Estados Unidos: Jhon Wiley \& Sons Ltd, 2007. 326 p.
	
	GUERRA, John Alexis; ZUNIGA, Maria Fernanda y RESTREPO, Felipe. Proyecto Iris. Trabajo de grado Ingeniero de Sistemas y Computación, Pereira: Universidad Tecnologia de Pereira, 2004
	
	MUÑOZ, Pablo Andrés. Máquinas de aprendizaje para reconocimiento de caracteres manuscritos. Tesis de Maestría en Ingeniería Eléctrica. Pereira: Universidad Tecnologica de Pereira, 2005
	
	RINCÓN, Jaime; LOAIZA, Johan Eric. Reconocimiento de palabras aisladas mediante redes neuronales sobre FPGA. Trabajo de grado Ingeniero Electricista. Pereira: Universidad Tecnologia de Pereira, 2008
	
	VALENCIA, Joan Mauricio; ABRIL Mauricio. Registro de transeúntes en tiempo real utilizando un sistema de visión artificial sobre un ambiente controlado. Trabajo de grado Ingeniero Electricista. Pereira: Universidad Tecnologia de Pereira, 2007
	
	ALVAREZ, Lorena. Acondicionamiento de Señales Bioeléctricas. Trabajo de grado Ingeniero Electricista. Pereira: Universidad Tecnologia de Pereira, 2007
	
	\clearpage
\end{document}
