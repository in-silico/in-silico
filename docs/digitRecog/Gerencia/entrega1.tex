\documentclass[a4paper, 12pt, oneside]{article}


% idioma
\usepackage[utf8]{inputenc}
\usepackage[spanish]{babel}

%tablas
\usepackage{booktabs}

%rotar tablas
\usepackage{rotating}

%color tablas
\usepackage{colortbl}



%espaciado
\usepackage{setspace}
\onehalfspacing
\setlength{\parindent}{0pt}
\setlength{\parskip}{2.0ex plus0.5ex minus0.2ex}


%margenes según n. icontec
\usepackage{vmargin}
\setmarginsrb           { 4.0cm}  % left margin
                        { 3.0cm}  % top margcm
                        { 2.0cm}  % right margcm
                        { 2.0cm}  % bottom margcm
                        {   0pt}  % head height
                        {0.0 cm}  % head sep
                        {   9pt}  % foot height
                        { 1.0cm}  % foot sep

% inserción url's notas de pie.
\usepackage{url}

% Paquetes de la AMS:
\usepackage{amsmath, amsthm, amsfonts}
\addto\captionsspanish{\def\refname{\textsc{Bibliografía}}}

\newcommand\portada{
	\begin{titlepage}
		\begin{center}
			{\large \bf PRIMERA ENTREGA DE GERENCIA DE PROYECTOS }
			\vfill
% 			{\large\bf PRESENTADO POR \par}
			{\large\bf SEBASTIÁN GÓMEZ GONZÁLEZ \par}
			{\large\bf SANTIAGO GUTIERREZ ALZATE \par}
			\vfill
			{\large\bf UNIVERSIDAD TECNOLÓGICA DE PEREIRA  \par}
			{\large\bf FACULTAD DE INGENIERÍAS \par}
			{\large\bf INGENIERÍA DE SISTEMAS Y COMPUTACIÓN \par}
			{\large\bf PEREIRA\par}
			{\large\bf SEPTIEMBRE DE 2010 \par}
		\end{center}
	\end{titlepage}
}


\begin{document}
\portada

	%\clearpage
	
	\begin{center}
	\section{Formulación del problema}
	\end{center}

	Los problemas visuales afectan a una gran parte de la población, tanto en Colombia\footnote{Según el DANE en su gran censo nacional realizado en el año 2005, se estima que el 6.3\% de la población colombiana sufre de alguna discapacidad. De estas personas con discapacidad se estima que el 43.4\% tienen dificultades para ver aun con lentes.} como a nivel mundial\footnote{La organización mundial de la salud estima que en el mundo hay 161 millones de personas con problemas visuales, de los cuales 37 millones son ciegos.}, estos traen consigo consecuencias devastadoras para las personas que los sufren, afectando su calidad de vida significativamente. Entre todos los problemas que sufren las personas con discapacidades visuales, uno de los más críticos es el del acceso a la información; esto debido a que la mayoría de la información los seres humanos reciben del entorno llega a través de los ojos. Los problemas para obtener información causan que las personas con discapacidades visuales se queden rezagadas con respecto a los demás miembros de la sociedad, impidiendo así que tengan las mismas oportunidades de vida que una persona con sus cinco sentidos intactos. Generalmente, esto causa un efecto de exclusión y de segregación en estas personas debido a que son vistas como una carga, tanto por si mismas como por las personas que los rodean. 

	A partir de la incidencia del problema, una forma de ayudar a que las personas con discapacidades visuales tengan una mejor calidad de vida es lograr que puedan tener acceso a la información que se encuentra en documentos escritos, tales como cartas, libros, y volantes. Esto puede ser logrado a través un sistema de reconocimiento óptico de caracteres que reconozca el texto escrito en estos documentos y le lea a la persona invidente lo que dicen. Sin embargo, para que sea una solución efectiva al problema, se requiere de un sistema que le permita a los discapacitados visuales acceder a la información escrita de una manera rápida y fácil, lo cual implica gran movilidad. Si bien existen múltiples sistemas de reconocimiento óptico de caracteres para computadores de escritorio, estos no poseen dicha movilidad, porque además del computador, también necesitan de un escáner para funcionar. Así pues, para que un sistema de este tipo pueda ser utilizado en múltiples lugares, se requiere de algún dispositivo que sea: pequeño, fácilmente transportable y capaz de hacer el trabajo en un tiempo razonable. Los teléfonos inteligentes parecen ser una buena alternativa, no obstante, existen grandes diferencias entre un computador de escritorio y un teléfono inteligente, entre ellas se pueden resaltar las siguientes:
	
	\begin{itemize}
	
	\item La memoria y la capacidad de procesamiento son muy reducidas en los teléfonos inteligentes con respecto a los computadores de escritorio.

	\item El tamaño de la memoria cache es mucho más pequeño en los teléfonos inteligentes. Como resultado de esto, los algoritmos diseñados para computadores de escritorio pueden ser muy lentos en los teléfonos inteligentes, pues se hicieron pensando en caches de varios mega-bytes.

	\item Un computador de escritorio normalmente tiene instaladas múltiples librerías de propósito general que son utilizadas por distintos programas, un teléfono inteligente, en cambio, solo provee las librerías más básicas.

	\end{itemize}

	La diferencia mas importante, sin embargo, es que en los computadores de escritorio las imágenes son obtenidas por medio de un escáner, logrando condiciones de iluminación muy buenas y uniformes, mientras en los teléfonos inteligentes las imágenes se obtienen por medio de una cámara fotográfica. Esta diferencia es muy importante para el reconocimiento óptico de caracteres, en especial si quien toma la foto es una persona invidente, ya que en este caso diversas variables como la inclinación del texto, la iluminación no uniforme, y las deformaciones elásticas del texto debido a la forma del papel en reposo, cobran más relevancia, pues la persona invidente podría tomar la fotografía inclinada, en el sentido contrario, o en un lugar con poca luminosidad.

	Para dar solución al problema de limitación en la capacidad de procesamiento de los teléfonos inteligentes se puede utilizar una de sus grandes ventajas: son por naturaleza dispositivos de alta conectividad. Cada vez los planes de telefonía son mas económicos y hay mas disponibilidad de redes inalámbricas, a las cuales los teléfonos inteligentes tienen acceso, en diferentes sitios públicos como institutos educativos y bibliotecas. Esta alta conectividad posibilita la construcción de un sistema de reconocimiento óptico de caracteres con una arquitectura cliente-servidor, en la cual el teléfono inteligente pueda enviar la imagen por internet y esta sea procesada por un servidor con una capacidad de computo mucho mayor. Este sistema tendrá las mismas ventajas de movilidad y rapidez deseadas. La capacidad de procesamiento de los celulares inteligentes también ha incrementado en los últimos años, lo que permitiría hacer parte del procesamiento en el celular, con el objetivo de reducir el ancho de banda requerido para enviar la imagen al servidor y por tanto el costo.

	Para solucionar los problemas asociados con la inclinación del texto que se puede producir en las imágenes tomadas por la persona invidente, se hace necesario un sistema de reconocimiento óptico de caracteres que pueda reconocer exitosamente fotos tomadas con distintos grados de inclinación o incluso en el sentido contrario.

	Después de una búsqueda en más de 20 artículos científicos relacionados al tema de reconocimiento óptico de caracteres, y de revisar la documentación técnica de varios OCRs libres, no se encontró un sistema de reconocimiento de caracteres por un medio óptico que haya sido desarrollado para ser usado por personas invidentes en teléfonos inteligentes, que sea de código libre y abierto, y que solucione los problemas de inclinación y luminosidad antes presentados. 	
	
	\clearpage

	\begin{center}
	\section{Justificación}
	\end{center}
	
	Además de esto, un sistema móvil para reconocer texto de documentos impresos presentaría los siguientes beneficios respecto de los sistemas tradicionales como las impresoras braille y los lectores basados en escáner:

	\begin{itemize} 

	\item Un menor costo, al no ser necesario adquirir un computador de escritorio, escáner o impresora braille.

	\item Portabilidad: ninguno de los sistemas tradicionales esta diseñado para ser llevado en todo momento por el usuario, dificultando el acceso al conocimiento e información que no se encuentre en el lugar físico en el que se encuentra alguno de estos sistemas.
	
	\item Una solución económica que pueda ejecutarse en un dispositivo móvil, sin necesidad de hardware adicional, haría posible que cada invidente pudiera tener su propio sistema de OCR personal, mientras que los otros sistemas usualmente (por su costo) son adquiridos por instituciones en cantidad limitada.

	\item Aportar al conocimiento: Al desarrollar este aplicativo se tendrán que hacer pruebas experimentales cuyos resultados aporten a otras investigaciones.
	\end{itemize}
	\clearpage

	\begin{center}
	\section{Misión, Visión, Objetivos y Metas}
	\end{center}
	
	\subsection{Misión}
	Ser una empresa líder en la asistencia tecnológica a las personas invidentes, brindándoles
	soluciones que les permitan integrarse a la sociedad usando tecnologías
	de la información y de comunicaciones.
	
	\subsection{Visión}
	Ser una empresa mundialmente reconocida en la investigación y desarrollo de aplicativos y servicios
	para móviles, relacionados con procesamiento de imágenes y visión por computador enfocada a la 
	asistencia a las personas invidentes.
	
	\subsection{Objetivo general}
	Desarrollar un sistema que usando un teléfono inteligente, le permita a las personas invidentes acceder
	a textos impresos tomándole una foto a estos textos. Este sistema le prestará el servicio de convertir
	la imagen a texto a los usuarios invidentes y sus teléfonos convertirán el texto a voz.
	
	\subsection{Objetivos específicos}
	\begin{itemize}
	\item Realizar un estudio de mercado, técnico y financiero de la idea de proyecto.
	\item Realizar una investigación científica de los algoritmos y técnicas para llevar a cabo este proyecto.
	\item Diseñar e implementar una aplicación con las especificaciones antes mencionadas.
	\item Buscar financiamiento en varias instituciones para proyectos de emprendimiento y
	 de impacto social.
	\item Buscar convenios con empresas prestadoras de servicios de celulares y con las instituciones
	 para personas invidentes en cada país para iniciar la implantación del servicio en el mercado.
	\end{itemize}
	
	\subsection{Metas}
	\begin{itemize}
	\item Realizar un estudio de mercado, técnico y financiero de la idea de proyecto usando
		como plazo no mas de 6 meses.
	\item Realizar una investigación científica de los algoritmos y técnicas para llevar a cabo este proyecto,
		escribir artículos de las investigaciones, revisando cada dos meses los avances y resultados de la
		investigación.
	\item Diseñar e implementar una aplicación con las especificaciones antes mencionadas, en los plazos
	definidos en la etapa de investigación.
	\item Buscar financiamiento en varias instituciones para proyectos de emprendimiento y
	 de impacto social, a las que se puedan aplicar y en los plazos establecidos por cada uno de estos
	 programas de financiamiento.
	\item Buscar convenios con empresas prestadoras de servicios de celulares ofreciendo el servicio
	 y los clientes por un determinado porcentaje de el plan tomado por la persona invidente. Primero
	 se hablará con las compañías celulares de Colombia y con el INCI (Instituto nacional para ciegos).
	\end{itemize}
	
	\clearpage
	\begin{center}
	\section{Indicadores}
	\end{center}
	
	\subsection{Indicadores de monitoréo}
	\begin{itemize}
		\item {\bf Velocidad de desarrollo}($M_1$): Para medir el avance de desarrollo se tomarán la cantidad
			de líneas de código $l$ y se dividirán entre el tiempo que ha transcurrido $t$.
			\[M_1 = \frac{l}{t}\]
		\item {\bf Avance de desarrollo}($M_2$): Para medir el avance de desarrollo se tomarán la cantidad
			de casos de uso terminados $C_f$ y se dividirán entre la cantidad de casos de uso que se deberían
			llevar $C_p$.
			\[M_2 = \frac{C_f}{C_p}\]
		\item {\bf Efectividad del aplicativo}($M_3$): Se define efectividad como la cantidad de caracteres(Letras)
			que se leen exitosamente a la persona invidente. Si $L_c$ son las letras leídas exitosamente y $L_t$
			son el total de letras entonces:
			\[M_3 = \frac{L_c}{L_t}\]	
		\item {\bf Búsqueda de financiamiento}($M_4$): Este es un indicador para comparar la cantidad adicional de
			financiamiento recibida por unidad de tiempo. Si $\Delta f$ es la cantidad de dinero adicional recibida
			y $\Delta t$ es la cantidad de tiempo en la que se recibió este financiamiento, entonces:
			\[M_4 = \frac{\Delta f}{\Delta t}\]
	\end{itemize}
	
	\subsection{Indicadores de impacto}
	\begin{itemize}
		\item {\bf Personas beneficiadas}($I_1$): Cantidad de personas beneficiadas $P_b$ sobre el total
			de personas afectadas por el problema (Personas invidentes) $P_i$.
			\[I_1 = \frac{P_b}{P_i}\]
		\item {\bf Reconocimiento}($I_2$): Una manera indirecta de medir el reconocimiento del proyecto por su
		 finalidad social es por la cantidad de premios $P$ que otorgados por este fin. Aunque solo sería útil
		 al principio.
			\[I_2 = \frac{P}{\Delta t}\]
		\item {\bf Calidad de vida de los usuarios}($I_3,I_4$): Al terminar el proyecto para ser usado comercialmente,
			se podrían hacer encuestas que ayuden a medir el mejoramiento de la calidad de vida de los usuarios.
			Si $S$ es el promedio de los salarios de los usuarios, un primer indicador que mida el aumento porcentual
			en sus salarios podría ser:
			\[I_3 = \frac{\Delta S}{S}\]
			Si $F$ es el promedio de una medida subjetiva de su nivel de felicidad, otro posible indicador sería:
			\[I_4 = \frac{\Delta F}{F}\]
	\end{itemize}
	
	\subsection{Indicadores de resultado}
	\begin{itemize}
		\item {\bf Costo/Beneficio monetario del proyecto}($R_1$): Si la cantidad de personas horas hombre $H$,
			el costo promedio de la hora $C_h$, y el nivel de ingresos de los $N$ beneficiarios aumenta en
			$I_3$(Ver definición en indicadores de impacto) puntos porcentuales. Se calcularía:
			\[R_1 = \frac{N \times I_3 \times S}{H \times C_h}\]
			E indicaría cuanto aumentaron los ingresos en general de los usuarios por cada peso invertido
			al proyecto.
		\item {\bf Crecimiento empresarial}($R_2$): Si en un periodo de tiempo $t$, entran unos ingresos $I$ y
			se tienen unos gastos $C$; Este indicador se da por:
			\[R_2 = \frac{I-C}{t}\]
		\item {\bf Incremento de usuarios}($R_3$): Si N es el número de usuarios, este indicador se calcularía:
			\[R_3 = \frac{\Delta N}{N}\]
			E indicaría un incremento porcentual de los usuarios del servicio.
	\end{itemize}
	\clearpage
	
	\begin{center}
	\section{Contextualización del proyecto}
	\end{center}
	
	\subsection{Relación con el plan de competitividad}
	La relación del proyecto con el plan regional de competitividad se puede ver el cuadro \ref{regional}. Note
	que los principales programas a los que el proyecto está relacionado tienen que ver con la investigación,
	ciencia, tecnología y creación de empresa. Estos son importantes para que la región crezca, y ya que toda
	región tiene sectores en los que es mas competente, Risaralda gracias a la Universidad tecnológica y a
	ParqueSoft ha mostrado que tiene un gran potencial en sectores de la informática y tecnología.
	
	\begin{sidewaystable}[h]	
		\caption{Relación del proyecto con el plan de competitividad}
		\begin{tabular}{ | p{4cm} | p{4cm} | p{14cm} | }
		\hline
		Objetivos & Proyecto priorizado & Fundamentación\\
		\hline					
		Innovación, investigación, ciencia y tecnología & Red de Nodos de Innovación, ciencia y tecnología & Los spin off y los spin offs universitarios son mencionados en el plan de competitividad de Risaralda como variables importantes en este objetivo, nuestro proyecto, que se desprende del proyecto de grado y es por tanto un spin off universitario, es un proyecto de investigación en el área de visión e inteligencia artificial. Ya que este tipo de software es muy nuevo, e incluso inexistente en nuestro país, seria un producto innovador en nuestra región. \\
		\hline
		Emprendimiento y desarrollo empresarial & Programa de emprendimiento y empresarismo & La creación de empleo es parte fundamental del plan de competitividad de Risaralda, nuestra proyecto, que empieza como un emprendimiento, puede brindar oportunidad de empleo a muchas personas de la región que se interesen en el campo de la inteligencia artificial y el reconocimiento de patrones. El proyecto también comprende la creación de una empresa dedicada al área de inteligencia artificial, algo acorde con este objetivo regional. \\
		\hline
		Fortalecimiento de sectores estratégicos & Ciudad digital y gobierno en línea & Ya que el sector de las tecnologías de la información y las comunicaciones es un sector priorizado en el plan de competitividad, nuestro proyecto, que involucra tanto información como el área de comunicaciones, puede hacer un aporte interesante a este sector. Así mismo, el programa gobierno en línea establece que debe haber una gran accesibilidad para posibilitar el acceso a la información a las personas invidentes, nuestro proyecto comparte este propósito y podría expandirlo para incluir también gran cantidad de documentos impresos que aun no se han digitalizado. \\
		\hline
		\end{tabular}
		\label{regional}
	\end{sidewaystable}

	\subsection{Relación con los planes del país}
	En el cuadro \ref{nacional} se muestra como el proyecto tiene también relación con los planes de
	la ciudad, del departamento y del país. Cabe destacar que este proyecto también tiene una
	fuerte relación con las metas del milenio establecidas por la organización de las naciones
	unidas (ONU).
	
	\begin{sidewaystable}[h]
		\caption{Relación del proyecto con los planes del país}
		\begin{tabular}{ | p{3cm} | p{8cm} | p{3cm} | p{3cm} | p{6cm} | }
		\hline
		Plan & Objetivo & Programa & Proyecto & Fundamentación \\
		\hline					
		Plan de Desarrollo Municipal Pereira “Región de Oportunidades” 2.007-2011 & “Cambiar las causas que generan las inequidades y retrasan el mejoramiento en las condiciones de vida de las personas, avanzando hacia un sistema complementario de asistencia pública integral y responsable con la generación de oportunidades para el desarrollo autónomo, bajo un esquema de corresponsabilidad social, armonizado con los compromiso de la humanidad en las metas del milenio, la visión Colombia 2019, el Plan Nacional de Desarrollo y el Programa de Gobierno avalado por la ciudadanía de Pereira en el voto de confianza asignado a la presente administración.” & Programa población prioritaria & Proyecto “Atención Sin Distinción” & Este proyecto involucra “atención a la población con limitaciones para ver, para caminar, para oír, para entender aprender, etc", lo que significa que la atención a las personas invidentes es uno de sus objetivos, nuestro proyecto es acorde con dicho objetivo. \\
		\hline
		Plan de Desarrollo Departamental “Risaralda Sentimiento de Todos” & "Los Ocho Objetivos del Milenio establecidos en los convenios internacionales y ratificados por Colombia, porque ningún grupo social podrá alcanzar metas mínimas de desarrollo humano sostenible sin acceso a servicios de educación, salud, seguridad alimentaria, recreación, vivienda digna, agua potable y saneamiento básico, salud ambiental y mejoramiento del ingreso, entre otros factores" & Línea Estratégica Equidad e Inclusión Social & Programa: La escuela un lugar para todos & Dado que una de las grandes dificultades para las personas invidentes a la hora de estudiar es el acceso a la información, nuestro proyecto puede ayudar a que estas personas accedan a la información escrita, que en muchos casos es la única información disponible, especialmente en los colegios de los municipios menos desarrollados. \\
		\hline
		Plan de Desarrollo Nacional 2.010-2014 "Prosperidad para Todos" & "Ser un país con prosperidad para todos: con más empleo, menor pobreza y más seguridad." & Formación de capital humano & Calidad en la formación impartida por el sistema educativo colombiano & La calidad de la educación depende del nivel de acceso a la información de quienes la reciben, nuestro proyecto busca facilitar este acceso a las personas invidentes. \\
		\hline
		\end{tabular}
		\label{nacional}
	\end{sidewaystable}
	
	\subsection{Relación sistemas de información}

	Nuestro proyecto se relaciona con SIPER, Sistema de información para el emprendimiento en Risaralda, ya que precisamente es un emprendimiento y por tanto este sistema de información nos puede ser útil para conectarnos con emprendimientos relacionados.

	\subsection{Relación área metropolitana}

	Nuestro proyecto se puede relacionar con el proyecto de emprendimiento del área metropolitana, por que ambos tienen que ver con emprendimiento en las universidades.

	\clearpage

	\begin{center}
	\section{Consideraciones adicionales}
	\end{center}

	El proyecto que se describe en este documento es muy acorde a los planes de desarrollo consultados, pero, especialmente, a los objetivos del milenio acordados por los 192 países miembros de las Naciones Unidas. Dichos objetivos no solo son mencionados en todos los planes de desarrollo, sino que son base fundamental de los mismos. El objetivo del milenio más relacionado con nuestro proyecto es el 2do: lograr la enseñanza primaria universal. Este objetivo no puede llevarse a cabo si los niños invidentes en los lugares más alejados no tienen acceso a información escrita, algo que pensamos podemos ayudar a solucionar con este proyecto.
	El proyecto también apunta al mejoramiento de la calidad de vida de las personas invidentes, al darles
	acceso al conocimiento que se encuentra en textos impresos. Las personas invidentes necesitan no solo
	poder comunicarse entre ellos, sino poderse comunicar con el resto de personas. Es por esto que el sistema
	de escritura tradicional Braille se desea remplazar lentamente con alternativas de alta tecnología
	que les permita a las personas invidentes gozar de una mayor inclusión en la sociedad, conseguir empleos
	y tener cada vez menos restricciones por su limitación.
	
	\clearpage
	\begin{center}
	\section{Bibliografía}
	\end{center}
	AGRAWAL, Mudit y DOERMANN, David. Voronoi++: A Dynamic Page Segmentation approach based on Voronoi and Docstrum features. INTERNATIONAL CONFERENCE ON DOCUMENT ANALYSIS AND RECOGNITION. (10: 26-29, julio, 2009: Barcelona, Spain). Memorias, 2009. p. 1011-1015
	
	O'GORMAN Lawrence. The Document Spectrum for Page Layout Analysis. \underline{En}: IEEE Transactions on Pattern Analysis And Machine Intelligence. Noviembre, 1993. vol. 15, no. 11, p. 1162-1173
	
	BREUEL, Thomas. The OCRopus Open Source OCR System. DOCUMENT RECOGNITION AND RETRIVAL. (15: 29-31, febrero, 2008: San Jose, Estados Unidos). Memorias, 2008, p. 68-150
	
	SHAFAITA Faisal; KEYSERSA, Daniel y BREUEL, Thomas. Efficient Implementation of Local Adaptative Thersholding Techniques Using Integral Images. DOCUMENT RECOGNITION AND RETRIVAL. (15: 29-31, febrero, 2008: San Jose, Estados Unidos). Memorias, 2008, p. 61-67
	
	MAO Song; AZRIEL, Rosenfelda y KANUNGOB, Tapas. Document structure analysis algorithms: A literature survey. DOCUMENT RECOGNITION AND RETRIVAL. (10: enero, 2003: San Jose, Estados Unidos). Memorias, 2003, p. 197-207
	
	SHAFAITA Faisal; KEYSERSA, Daniel y BREUEL, Thomas. Performance Evaluation and Benchmarking of Six-Page segmentation Algorithms. \underline{En}: IEEE Transactions on Pattern Analysis And Machine Intelligence. Junio, 2008. vol. 30, no. 6, p. 941-954
	
	KISE, Koichi; SATO Akinori y MATSUMOTO, Keinosuke. Document Image Segmentation as Selection of Voronoi Edges. IEEE COMPUTER SOCIETY: CONFERENCE ON COMPUTER VISION AND PATTERN RECOGNITION (66: 17-19, junio, 1997: San Juan, Puerto Rico). Memorias, 1997, p. 32-39
	
	WONG, Kwan; CASEY, Richard y WAHL, Friedrich. Document Analysis System. \underline{En}: IBM Journal of Research and Development. Noviembre, 1982. vol. 26, no. 6, p. 647-656
	
	Ray Smith. An Overview of the Tesseract OCR Engine. IEEE COMPUTER SOCIETY: INTERNATIONAL CONFERENCE ON DOCUMENT ANALYSIS AND RECOGNITION (9: 23-26, septiembre, 2007: Curitiba, Paraná, Brazil). Memorias, 2007, p. 629-633
	
	ZHOU, Steven; SYED, Gilani y WINKLER, Stefan. Open Source OCR Framework Using Mobile Devices. MULTIMEDIA ON MOBILE DEVICES (1: 28-29 enero, 2008: San Jose, California, Estados Unidos). Memorias, 2008, vol. 6821, p. 682104.1-682104.6
	
	SENDA, Shuji, et al. Camera-Typing Interface for Ubiquitous Information Services. IEEE ANNUAL CONFERENCE ON PERVASIVE COMPUTING AND COMMUNICATIONS (2: 14-17, marzo, 2004: Orlando, Florida, Estados Unidos). Memorias, 2004, p. 366-372
	
	SMITH, Ray. Progress in Camera-Based Document Image Analysis. INTERNATIONAL CONFERENCE ON DOCUMENT ANALYSIS AND RECOGNITION (7: 3-6, agosto, 2003: Edimburgo, Escocia). Memorias, 2003, p. 606-617
	
	HU, Ming-Kuei. Visual pattern recognition by moment invariants. \underline{En}: IRE Transactions on Information Theory. Febrero, 1962, vol. 8, no. 2, p. 179-187
	
	SEUL, Machael , O'GORMAN Lawrence y SAMMON, Michael J. Practical Algorithms for Image Analysis. Nueva York: Cambridge University Press, 2000. 301 p.
	
	FLUSSER Jan, SUK, Tomas Suk y ZITOVA, Barbara. Moments and Moment Invariants in Pattern Recognition. Sussex, Reino Unido: Jhon Wiley \& Sons Ltd, 2009. 291 p.
	
	ALPAYDIN, Ethem. Introduction to Machine Learning. Estados Unidos: MIT Press, 2004. 415 p.
	
	CHERIET, Mohamed; KHARMA, Nawwaf y LIU Cheng-Lin. Character recognition systems: Guide for students and practitioners. New Jersey, Estados Unidos: Jhon Wiley \& Sons Ltd, 2007. 326 p.
	
	GUERRA, John Alexis; ZUNIGA, Maria Fernanda y RESTREPO, Felipe. Proyecto Iris. Trabajo de grado Ingeniero de Sistemas y Computación, Pereira: Universidad Tecnologia de Pereira, 2004
	
	MUÑOZ, Pablo Andrés. Máquinas de aprendizaje para reconocimiento de caracteres manuscritos. Tesis de Maestría en Ingeniería Eléctrica. Pereira: Universidad Tecnologica de Pereira, 2005
	
	RINCÓN, Jaime; LOAIZA, Johan Eric. Reconocimiento de palabras aisladas mediante redes neuronales sobre FPGA. Trabajo de grado Ingeniero Electricista. Pereira: Universidad Tecnologia de Pereira, 2008
	
	VALENCIA, Joan Mauricio; ABRIL Mauricio. Registro de transeúntes en tiempo real utilizando un sistema de visión artificial sobre un ambiente controlado. Trabajo de grado Ingeniero Electricista. Pereira: Universidad Tecnologia de Pereira, 2007
	
	ALVAREZ, Lorena. Acondicionamiento de Señales Bioeléctricas. Trabajo de grado Ingeniero Electricista. Pereira: Universidad Tecnologia de Pereira, 2007
	
	\clearpage
\end{document}
