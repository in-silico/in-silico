\documentclass[a4paper, 12pt, oneside]{article}


% idioma
\usepackage[utf8]{inputenc}
\usepackage[spanish]{babel}

%tablas
\usepackage{booktabs}

%rotar tablas
\usepackage{rotating}

%color tablas
\usepackage{colortbl}



%espaciado
\usepackage{setspace}
\onehalfspacing
\setlength{\parindent}{0pt}
\setlength{\parskip}{2.0ex plus0.5ex minus0.2ex}


%margenes según n. icontec
\usepackage{vmargin}
\setmarginsrb           { 4.0cm}  % left margin
                        { 3.0cm}  % top margcm
                        { 2.0cm}  % right margcm
                        { 2.0cm}  % bottom margcm
                        {   0pt}  % head height
                        {0.0 cm}  % head sep
                        {   9pt}  % foot height
                        { 1.0cm}  % foot sep

% inserción url's notas de pie.
\usepackage{url}

% Paquetes de la AMS:
\usepackage{amsmath, amsthm, amsfonts}
\addto\captionsspanish{\def\refname{\textsc{Bibliografía}}}

\newcommand\portada{
	\begin{titlepage}
		\begin{center}
			{\large \bf ESTUDIO DE MERCADO GERENCIA DE PROYECTOS}
			\vfill
% 			{\large\bf PRESENTADO POR \par}
			{\large\bf SEBASTIÁN GÓMEZ GONZÁLEZ \par}
			{\large\bf SANTIAGO GUTIERREZ ALZATE \par}
			\vfill
			{\large\bf UNIVERSIDAD TECNOLÓGICA DE PEREIRA  \par}
			{\large\bf FACULTAD DE INGENIERÍAS \par}
			{\large\bf INGENIERÍA DE SISTEMAS Y COMPUTACIÓN \par}
			{\large\bf PEREIRA\par}
			{\large\bf SEPTIEMBRE DE 2010 \par}
		\end{center}
	\end{titlepage}
}


\begin{document}
\portada

	%\clearpage
	
	\begin{center}
	\section{Descripción del problema}
	\end{center}

	Los problemas visuales afectan a una gran parte de la población, tanto en Colombia\footnote{Según el DANE en su gran censo nacional realizado en el año 2005, se estima que el 6.3\% de la población colombiana sufre de alguna discapacidad. De estas personas con discapacidad se estima que el 43.4\% tienen dificultades para ver aun con lentes.} como a nivel mundial\footnote{La organización mundial de la salud estima que en el mundo hay 161 millones de personas con problemas visuales, de los cuales 37 millones son ciegos.}, estos traen consigo consecuencias devastadoras para las personas que los sufren, afectando su calidad de vida significativamente. Entre todos los problemas que sufren las personas con discapacidades visuales, uno de los más críticos es el del acceso a la información; esto debido a que la mayoría de la información los seres humanos reciben del entorno llega a través de los ojos. Los problemas para obtener información causan que las personas con discapacidades visuales se queden rezagadas con respecto a los demás miembros de la sociedad, impidiendo así que tengan las mismas oportunidades de vida que una persona con sus cinco sentidos intactos. Generalmente, esto causa un efecto de exclusión y de segregación en estas personas debido a que son vistas como una carga, tanto por si mismas como por las personas que los rodean. 


	Efectos directos del problema:
	\begin{itemize}
	\item Falta de accesibilidad a la información en medios impresos como los libros, cartas, documentos legales, entre otros.
	\end{itemize}

	Efectos indirectos:
	\begin{itemize}
	\item Imposibilidad de acceder a gran parte del conocimiento actualizado, que se encuentra en los libros, pero que por derechos de autor no se puede conseguir con facilidad en medios digitales.
	\item Imposibilidad de leer un documento legal antes de firmarlo. En este sentido, la persona invidente usualmente debe confiar plenamente en la persona que le lee el documento; esto les puede causar luego problemas legales.
	\end{itemize}
	
	Causas directas:
	\begin{itemize}
	\item Una gran parte de la información con la que tenemos contacto a diario se encuentra en medios visuales.
	\item No se encuentra casi información disponible en medios utilizables por personas invidentes y no se cuenta muchas veces con tecnología que facilite el acceso a la información en medios no accesibles para invidentes.
	\end{itemize}

	Causas indirectas:
	\begin{itemize}
	\item Gran parte de la información disponible en nuestro medio, se ha creado pensando únicamente en las mayorías (Personas sin discapacidades visuales).
	\item Es muy costoso proveer toda la información en medios accesibles tanto para las personas que pueden ver como para las personas que no pueden.
	\end{itemize}



	\begin{center}
	\section{Descripción de la situación actual y esperada}
	\end{center}

	Situación actual
	\begin{itemize}
	\item En este momento, para las personas invidentes es difícil acceder al conocimiento actualizado al que tienen acceso las personas no invidentes. 
	\item Si no se toman medidas las personas invidentes se seguirán considerando una carga para la sociedad, tendrán mas baja autoestima y pocos sueños con un futuro mejor. 
	\item En este momento, las personas invidentes pueden acceder al conocimiento en libros en Braille, que son muy costosos, pesados y difíciles de encontrar; O pueden utilizar la tecnología para acceder al conocimiento en formatos digitales a través de lectores de pantalla, y asumiendo que el contenido es accesible (Hay muchos formatos no accesibles como los plugins de Flash, texto en imágenes, entre otros). También se tiene acceso a libros impresos a través de programas OCR que funcionan con un escaner y un computador, que no son portátiles y son muy costosos (Solo unas pocas bibliotecas lo tienen).
	\item Según el INCI (Instituto Nacional de Ciegos), en Colombia las personas invidentes no cuentan con ninguna tecnología personal para el acceso al texto impreso.
	\item Según el DANE en su gran censo nacional realizado en el año 2005, se estima que el 6.3\% de la población colombiana sufre de alguna discapacidad. De estas personas con discapacidad se estima que el 43.4\% tienen dificultades para ver aun con lentes. 
	\item La organización mundial de la salud estima que en el mundo hay 161 millones de personas con problemas visuales, de los cuales 37 millones son ciegos. 
	\end{itemize}

	Situación actual
	\begin{itemize}
	\item Las personas invidentes deberían tener una mayor igualdad al acceso a la información actualizada con respecto a las personas no invidentes.
	\item Si se toman medidas al respecto, las personas invidentes podrán
desempeñar roles importantes en nuestra sociedad, lo que les mejoraría la autoestima y generaría ganas de soñar con un futuro alcanzable y mejor.
	\item Dado que es costoso publicar el conocimiento en 2 formatos diferentes (Para personas invidentes y no invidentes), una buena solución debería permitir a las personas invidentes acceder al conocimiento en libros hechos para personas no invidentes. Debería además ser económica y portátil, para permitir leer un libro en diferentes lugares y no tener que ir a un lugar específico para acceder a este conocimiento.
	\item Se esperaría que al hacer un proyecto tecnológico para brindar acceso al texto impreso en un dispositivo móvil, las personas invidentes puedan acceder a este conocimiento en libros impresos desde sus casas, lugares de trabajo, bibliotecas o institutos educativos; A un muy bajo costo o de manera gratuita si ya disponen de un celular con las características apropiadas.
	\end{itemize}

	\clearpage

	\begin{center}
	\section{Diagnóstico de problema o necesidad}
	\end{center}

	Población afectada por el problema o necesidad

	%IMAGEN 1

	Características sociales de los habitantes directamente afectados por el problema

Las personas invidentes, por tener menos oportunidades que el resto de las personas, tienden a no ser personas con ingresos muy elevados. Sin embargo el estado Colombiano tiene el deber de brindar una diferencia positiva para que las personas con algún tipo de discapacidad puedan estar en condiciones un poco mas equitativas con respecto al resto de la población.

	%IMAGEN 2

Zona o área afectada por el problema o necesidad

En todo el mundo hay personas invidentes, aunque la población objetivo del proyecto es en un principio la población de personas invidentes de nuestro país Colombia. Se estima que en Colombia hay al menos unas 100000 personas que tienen grandes dificultades para leer aun con lentes.

	%IMAGEN 3

Estudio del INCI, muestra el ICV(Índice de calidad de vida) por departamento de hogares comunes (En rojo) y de hogares con LV (Limitados Visuales, en azul).

Delimitación del problema o necesidad

El problema a nivel macro consiste en la desigualdad de condiciones y de oportunidades para las personas con discapacidades visuales. Sin embargo, se delimita el problema a la dificultad de acceder a la información y al conocimiento en texto impreso.

Problema general 

Las personas con discapacidades visuales, no tienen las mismas oportunidades que tienen las personas sin discapacidades visuales. Trayendo efectos 
sociales, económicos, psicológicos y emocionales para las personas invidentes, a las que además, muchas veces se les considera como una carga para la sociedad.

Delimitación

La discapacidad para leer libros impresos, no permite muchas veces que las personas tengan acceso a conocimiento actualizado. Es un tipo de analfabetismo que afecta exclusivamente a las personas con discapacidades visuales pero tiene todos los efectos devastadores del analfabetismo que conocemos tradicionalmente.


Población objetivo

Proyecto Piloto 

En la primera etapa del proyecto se pretende probar la efectividad del mismo en colegios con niños invidentes, ya que aquí es uno de los lugares en los que mas se necesita acceso al conocimiento en los libros.

Proyecto final 

El objetivo a largo plazo es que las personas invidentes puedan acceder a los beneficios del proyecto, de manera individual o a través de alguna institución que ayude a las personas invidentes en el país de origen.

\clearpage

	\begin{center}
	\section{Objetivos}
	\end{center}

A partir de la incidencia del problema, una forma de ayudar a que las personas con discapacidades visuales tengan una mejor calidad de vida es lograr que puedan tener acceso a la información que se encuentra en documentos escritos, tales como cartas, libros, y volantes. Un sistema que solucionase este problema, le posibilitaría a las personas con discapacidades visuales mejorar enormemente su calidad de vida, ya que les permitiría una mejor educación, al poder acceder a los mismos libros de texto que el resto de la sociedad, y no solo, como actualmente sucede, a los pocos libros que se encuentran disponibles en formato braille.

	%IMAGEN 4

\clearpage

	\begin{center}
	\section{Listado de opciones alternativas}
	\end{center}

Para dar una solución efectiva al problema se requiere de algún sistema que le permita a los discapacitados visuales acceder a la información escrita de una manera rápida y fácil, lo cual implica gran movilidad. Si bien existen sistemas de reconocimiento óptico de caracteres para computadores de escritorio, estos no poseen dicha movilidad, porque además del computador, también necesitan de un escáner para funcionar. Así pues, para que un sistema de este tipo pueda ser utilizado en múltiples lugares, se requiere de algún dispositivo que sea: pequeño, fácilmente transportable y lo suficientemente poderoso para hacer el trabajo en un tiempo razonable. Los celulares comunes cumplen las primeras dos características, pero no la tercera, ya que su poder de procesamiento es varias veces inferior al de un computador común. Sin embargo, una nueva generación de celulares, que se denomina la de los celulares inteligentes, cumple esta tercer característica, ya que su poder de computo es comparable con el de un computador portátil. Así pues, si existiese un sistema de reconocimiento óptico de caracteres diseñado para estos celulares, este podría utilizarse en un sistema que permitiese reconocer texto de imágenes adquiridas por medio de la cámara de estos celulares, que lograría mejorar la capacidad de acceso a la información de las personas con discapacidades visuales. Además de esto, un sistema móvil para reconocer texto de documentos impresos presentaría los siguientes beneficios respecto de los sistemas tradicionales como las impresoras braille y los lectores basados en escáner:

	\begin{itemize} 
	\item Un menor costo, al no ser necesario adquirir un computador de escritorio, escáner o impresora braille.
	\item Portabilidad: ninguno de los sistemas anteriormente mencionados esta diseñado para ser llevado en todo momento por el usuario, dificultando el acceso al conocimiento e información que no se encuentre en el lugar físico en el que se encuentra alguno de estos sistemas.
	\item Una solución económica que pueda ejecutarse en un dispositivo móvil, sin necesidad de hardware adicional, haría posible que cada invidente pudiera tener su propio sistema de OCR personal, mientras que los otros sistemas usualmente (por su costo) son adquiridos por instituciones en cantidad limitada.
	\item Aportar al conocimiento: Al desarrollar este aplicativo se tendrán que hacer pruebas experimentales cuyos resultados aporten a otras investigaciones.
	\end{itemize}

\clearpage

	\begin{center}
	\section{Estudio de mercado}
	\end{center}

	Antes de hablar del resto del estudio de mercado veamos algunas estadísticas interesantes de la población de personas con limitaciones visuales.

	%IMAGEN 5
	Estudio del INCI, cantidad de estudiantes con limitaciones visuales.

	Estas cifras son importantes porque muestras una tendencia de la población de personas invidentes por ingresar cada vez mas a instituciones educativas, lo cual es una buena noticia, pero muestra la necesidad de elementos que les permita acceder a la información en libros impresos de una manera ágil.

	En la siguiente tabla se muestra como la población de personas con discapacidades visuales que están en un proceso educativo, se dividen en las diferentes etapas de formación. De esto se puede observar que el porcentaje mas alto de la población esta en la secundaria y siguiente está la primaria. Un porcentaje muy bajo de estos chicos entra a un nivel de educación superior. Otro problema que se observa es como pueden haber mas personas en la secundaria que en la primaria y niveles inferiores; Una de las posibles causas de esto es la división que se da todavía entre las personas con y sin discapacidades visuales. Es posible que en estas materias básicas, por la necesidad de aprender a leer, pero las diferencias tan grandes entre como las personas lo hacer, no se puedan dar los mismos ritmos de aprendizaje y esto cause la segregación vista en las estadísticas.

	%IMAGEN 6
	
	%IMAGEN 7

	%IMAGEN 8

Un detalle preocupante de estas cifras es el alto índice de analfabetismo, un 31.6\% de las personas con LV son analfabetas. Lo cual es el problema que se desea atacar en este momento. 

	%IMAGEN 9

	El otro detalle, para poder encontrar un precio justo que las personas estén dispuestas a pagar, es tener en cuenta el estrato socio-económico de la población. En la tabla anteriormente presentada se muestra que el porcentaje de personas con LV en un estrato mayor al 3 no supera al 12\%. Esta información nos muestra la realidad económica de las personas con discapacidades visuales en nuestro país, es importante tener todo esto en cuenta antes de continuar el estudio de mercado. 

	%IMAGEN 10
	Fuente: Cifras del DANE

Otras cifras laborales:

	\begin{itemize} 
	\item El 90\% de la población de invidentes no está capacitada para trabajar.
	\item En este momento hay 4 proyectos productivos promovidos por personas invidentes, esta puede ser una buena solución para la falta de empleo tan grande en el país, aunque solo 4 proyectos de emprendimiento en todo el país no es una cifra muy alentadora, nos da esperanzas; Que podrían hacer estas personas con las oportunidades apropiadas. 
	\end{itemize}

	Dadas todas las cifras anteriores, parece ser que la mejor alternativa es ir directamente a las instituciones donde hallas personas invidentes, y ofrecerles una solución para el acceso a el conocimiento en libros impresos usando algún dispositivo móvil económicos con una cámara de buena resolución y usando software a través de una red de datos. Esta sería la solución mas económica para instituciones, ya que con una única red local de datos podrían prestar servicio a todas las personas dentro de la institución sin tener que pagar ningún plan de datos.

Costos tentativos (Mensual)

Equipos de cómputo \$1.000.000,00
Plan de internet para servidor \$10.000.000,00
Salarios \$20.000.000,00
Otros \$10.000.000,00
Total \$41.000.000,00

Ingresos tentativos (Mensual)

Supongamos que un 10\% de la población de personas invidentes usara nuestro producto, contando a las personas independientes que desean adquirir el servicio y a las personas que se benefician de este servicio a través de alguna institución que también halla adquirido el servicio.

Cantidad de personas con el servicio 20.000
Costo promedio del servicio por persona \$3.000,00
Total \$60.000.000,00

Hay que tener en cuenta que este costo por persona tentativo dado, es un costo promedio, esto porque la idea es que las instituciones paguen menos por persona que las personas individuales que quieran adquirir el servicio.
	
	\clearpage

\clearpage

	\begin{center}
	\section{Bibliografía}
	\end{center}

	AGRAWAL, Mudit y DOERMANN, David. Voronoi++: A Dynamic Page Segmentation approach based on Voronoi and Docstrum features. INTERNATIONAL CONFERENCE ON DOCUMENT ANALYSIS AND RECOGNITION. (10: 26-29, julio, 2009: Barcelona, Spain). Memorias, 2009. p. 1011-1015
	
	O'GORMAN Lawrence. The Document Spectrum for Page Layout Analysis. \underline{En}: IEEE Transactions on Pattern Analysis And Machine Intelligence. Noviembre, 1993. vol. 15, no. 11, p. 1162-1173
	
	BREUEL, Thomas. The OCRopus Open Source OCR System. DOCUMENT RECOGNITION AND RETRIVAL. (15: 29-31, febrero, 2008: San Jose, Estados Unidos). Memorias, 2008, p. 68-150
	
	SHAFAITA Faisal; KEYSERSA, Daniel y BREUEL, Thomas. Efficient Implementation of Local Adaptative Thersholding Techniques Using Integral Images. DOCUMENT RECOGNITION AND RETRIVAL. (15: 29-31, febrero, 2008: San Jose, Estados Unidos). Memorias, 2008, p. 61-67
	
	MAO Song; AZRIEL, Rosenfelda y KANUNGOB, Tapas. Document structure analysis algorithms: A literature survey. DOCUMENT RECOGNITION AND RETRIVAL. (10: enero, 2003: San Jose, Estados Unidos). Memorias, 2003, p. 197-207
	
	SHAFAITA Faisal; KEYSERSA, Daniel y BREUEL, Thomas. Performance Evaluation and Benchmarking of Six-Page segmentation Algorithms. \underline{En}: IEEE Transactions on Pattern Analysis And Machine Intelligence. Junio, 2008. vol. 30, no. 6, p. 941-954
	
	KISE, Koichi; SATO Akinori y MATSUMOTO, Keinosuke. Document Image Segmentation as Selection of Voronoi Edges. IEEE COMPUTER SOCIETY: CONFERENCE ON COMPUTER VISION AND PATTERN RECOGNITION (66: 17-19, junio, 1997: San Juan, Puerto Rico). Memorias, 1997, p. 32-39
	
	WONG, Kwan; CASEY, Richard y WAHL, Friedrich. Document Analysis System. \underline{En}: IBM Journal of Research and Development. Noviembre, 1982. vol. 26, no. 6, p. 647-656
	
	Ray Smith. An Overview of the Tesseract OCR Engine. IEEE COMPUTER SOCIETY: INTERNATIONAL CONFERENCE ON DOCUMENT ANALYSIS AND RECOGNITION (9: 23-26, septiembre, 2007: Curitiba, Paraná, Brazil). Memorias, 2007, p. 629-633
	
	ZHOU, Steven; SYED, Gilani y WINKLER, Stefan. Open Source OCR Framework Using Mobile Devices. MULTIMEDIA ON MOBILE DEVICES (1: 28-29 enero, 2008: San Jose, California, Estados Unidos). Memorias, 2008, vol. 6821, p. 682104.1-682104.6
	
	SENDA, Shuji, et al. Camera-Typing Interface for Ubiquitous Information Services. IEEE ANNUAL CONFERENCE ON PERVASIVE COMPUTING AND COMMUNICATIONS (2: 14-17, marzo, 2004: Orlando, Florida, Estados Unidos). Memorias, 2004, p. 366-372
	
	SMITH, Ray. Progress in Camera-Based Document Image Analysis. INTERNATIONAL CONFERENCE ON DOCUMENT ANALYSIS AND RECOGNITION (7: 3-6, agosto, 2003: Edimburgo, Escocia). Memorias, 2003, p. 606-617
	
	HU, Ming-Kuei. Visual pattern recognition by moment invariants. \underline{En}: IRE Transactions on Information Theory. Febrero, 1962, vol. 8, no. 2, p. 179-187
	
	SEUL, Machael , O'GORMAN Lawrence y SAMMON, Michael J. Practical Algorithms for Image Analysis. Nueva York: Cambridge University Press, 2000. 301 p.
	
	FLUSSER Jan, SUK, Tomas Suk y ZITOVA, Barbara. Moments and Moment Invariants in Pattern Recognition. Sussex, Reino Unido: Jhon Wiley \& Sons Ltd, 2009. 291 p.
	
	ALPAYDIN, Ethem. Introduction to Machine Learning. Estados Unidos: MIT Press, 2004. 415 p.
	
	CHERIET, Mohamed; KHARMA, Nawwaf y LIU Cheng-Lin. Character recognition systems: Guide for students and practitioners. New Jersey, Estados Unidos: Jhon Wiley \& Sons Ltd, 2007. 326 p.
	
	GUERRA, John Alexis; ZUNIGA, Maria Fernanda y RESTREPO, Felipe. Proyecto Iris. Trabajo de grado Ingeniero de Sistemas y Computación, Pereira: Universidad Tecnologia de Pereira, 2004
	
	MUÑOZ, Pablo Andrés. Máquinas de aprendizaje para reconocimiento de caracteres manuscritos. Tesis de Maestría en Ingeniería Eléctrica. Pereira: Universidad Tecnologica de Pereira, 2005
	
	RINCÓN, Jaime; LOAIZA, Johan Eric. Reconocimiento de palabras aisladas mediante redes neuronales sobre FPGA. Trabajo de grado Ingeniero Electricista. Pereira: Universidad Tecnologia de Pereira, 2008
	
	VALENCIA, Joan Mauricio; ABRIL Mauricio. Registro de transeúntes en tiempo real utilizando un sistema de visión artificial sobre un ambiente controlado. Trabajo de grado Ingeniero Electricista. Pereira: Universidad Tecnologia de Pereira, 2007
	
	ALVAREZ, Lorena. Acondicionamiento de Señales Bioeléctricas. Trabajo de grado Ingeniero Electricista. Pereira: Universidad Tecnologia de Pereira, 2007
	
	\clearpage

\end{document}
