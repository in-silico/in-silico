\documentclass[a4paper, 11pt, oneside]{article}

% idioma
\usepackage[utf8]{inputenc}
\usepackage[spanish]{babel}

%tablas
\usepackage{booktabs}

%rotar tablas
\usepackage{rotating}

%color tablas
\usepackage{colortbl}

%espaciadorecursos
\usepackage{setspace}
\onehalfspacing
\setlength{\parindent}{0pt}
\setlength{\parskip}{2.0ex plus0.5ex minus0.2ex}

%margenes según n. icontec
\usepackage{vmargin}
\setmarginsrb           { 4.0cm}  % left margin
                        { 3.0cm}  % top margcm
                        { 2.0cm}  % right margcm
                        { 2.0cm}  % bottom margcm
                        {   0pt}  % head height
                        { 0.0cm}  % head sep
                        {   9pt}  % foot height
                        { 1.0cm}  % foot sep

% inserción url's notas de pie.
\usepackage{url}

\begin{document}

\section{Definiciones}

\subsection{Visión}

Es un conjunto de ideas generales, que marca lo que una empresa quiere ser y es. En general 
no debe expresarse en términos cuantitativos, sino motivacionales.

\subsection{Objetivos estratégicos}

Vínculo entre misión y visión. Contemple distintas alternativas para llegar a lo estipulado
en la visión. Relacionasdos con desarrollos en mercadeo, tecnología, info, servicio al cliente 
y talento humano, entre otros.

\subsection{Estrategias}

Acciones que deben realizarse para mantener y soportar los logros de la org y de cada unidad de
trabajo para poder hacer realidad los resultados esperados al momento de definir los proyectos 
estratégicos.

\begin{itemize}
	\item Metas: Cuantificación de los objetivos
	\item Índice: Relación entre cantidades x/y
	\item Indicador: Mide el comportamiento de dos o mas cantidades a través del tiempo
\end{itemize}

Indicadores son de 3 naturalezas:

\begin{itemize}
	\item Seguimiento o monitoreo
	\item Evaluación: A que costo de logro la meta
	\item Impacto: Que se logró con el cumplimiento de la meta
\end{itemize}

\subsection{Proyecto}

Un proyecto es una planificación que se hacer con el fin de lograr unos objetivos mediante un
conjunto de actividades relacionadas y coordinadas entre si, busca la creación de bienes o 
servicios con el potencial de resolver problemas. Es una secuencia bien definida de actividades
con un principio y un final, y se centra en lograr un objetivo claro, mateniendo la calidad
con un límite de costos y de tiempo. Es una solución inteligente para resover un problema.

Todo proyecto combina la utilización de recursos humanos, técnicos, financieros y comerciales,
de acuerdo con su diseño y conceptualización.

\section{Proyección de mercado}

\subsection{Econométrico}

Medida de lo económico, compara la oferta con la demanda.

\[Q_0 = Q_d + \Delta s + x - m \]

\begin{itemize}
	\item $Q_0$: es la cantidad ofrecida, función de precio, cotro de los factores, sutitutos y otras variables $f_1(P,CA,C,O)$.
	\item $Q_d$: Cantidad demandada, es función del precio del producto de productos sustitutos y otras variables $f_2(P,NA,PS,O)$
	\item $\Delta S$: Variación en el inventario, en función de cantidad de productos terminados, precio y precio esperado $f_3(\Delta Q, p, pe)$
	\item $m$: Es el nivel de importaciones en fución del precio de importación, el precio y otras var $f_4(P, Pm, O)$
	\item $x$: Nivel de exportaciones función del precio de exportación, precio y otras variables $f_5(Px,P,O)$.
\end{itemize}

\subsection{Modelo insumo - producto}

Costo y disponibilidad de los insumos, para determinar la disponibilidad de los mismos. Analizar la dependencia
de la empresa de otras, solucionando muhcas veces las dependencias con desarrollo vertical (Produciendo ellos
mismos sus insumos).

\subsection{Series de tiempo}

Analisis histórico, vigente y de la situación proyectada. Levanto la información que se posee del producto en el
pasado, y se pondera la información para hallar situación proyectada. Se utiliza cuando el comportamiento que se puede
presentar en el futuro se puede precisar por lo sucedido en el pasado siempre que esté disponible la información histórica
de manera confiable y completa. Se requiere:

\begin{itemize}
	\item Identificación de una o mas variables que pueden influir sobre la demanda.
	\item Seleccionar la relación entre variables causales y comportamiento del mercado (Equación matemática).
	\item Validación del modelo-pronóstico
\end{itemize}

\subsubsection{Componentes de las series de tiempo}

\begin{itemize}
	\item Componente estacional:
	Es el que exhibe variaciones que se repiten periódicamente. En países desarrollados se asicia a las etaciones
	climáticas, en nuestros paises se asocia a épocas en las que se espera un comportamiento similar.
	\item Componente de tendencia:
	Se relaciona con el crecimiento o declinación en el largo plazo de la variable estudiada.
	\item Componente cíclico:
	Divergencia entre la líne de tendencia proyectada y el valor real.
	\item Componente sistémico:
	Corresponde a la aparición de eventos aleatorios que afectan a la variable estudiada.
\end{itemize}

El modelo de Derviciotis permite expresar la variable como el producto de los componentes.

\subsection {Método de promedios móviles}

Es una ventana que se mueve por cada periodo, y en cada ventana se calcula un promedio.

\begin{table}
	\begin{center}
	\caption{Relación del proyecto con los planes del país}
	\begin{tabular}{ | p{1.5cm} | p{1.5cm} | p{1.5cm} | p{1.5cm} | p{1.5cm} | }
	\hline
	Año & Invierno & Primavera & Verano & Otoño\\
	\hline
	1977 & 0 & 0 & 1,39 & 0,28\\
	\hline
	1978 & 1,14 & 1,23 & 1,33 & 0,33\\
	\hline
	1979 & 1,02 & 1,36 & 1,27 & 0,33\\
	\hline
	1980 & 0,93 & 1,49 & 1,38 & 0,16\\
	\hline
	1981 & 0,91 & 1,2 & 1,53 & 0,43\\
	\hline
	1982 & 0,82 & 1,46 & 1,45 & 0,12\\
	\hline
	1983 & 0,98 & 1,29 & 1,54 & 0,35\\
	\hline
	1984 & 0,78 & 1,22 & 1,49 & 0,53\\
	\hline
	1985 & 0,87 & 1,13 & 1,53 & 0,43\\
	\hline
	1986 & 0,85 & 1,24 & 1,45 & 0,48\\
	\hline
	1987 & 0,81 & 1,18 & 0 & 0\\
	\hline
	Total & 9,11 & 12,8 & 14,36 & 3,44\\
	\hline
	Promedio & 0,911 & 1,28 & 1,436 & 0,344\\
	\hline
	\end{tabular}
	\end{center}
	\label{nacional}
\end{table}

\subsection{Tener en cuenta para el estudio de mercado}

En un estudio de mercado se tiene un mercado potencial, uno real y uno efectivo.

\begin{itemize}
	\item ¿Que se quiere estudiar?
	\item ¿Quienes conforman el mercado (Público objetivo)?
	\item Calcular el tamaño de la muestra con el mercado efectivo.
\end{itemize}

Para hallar la demanda, hay cuatro formas:

\begin{itemize}
	\item Encuestas
	\item Análisis multivariable
	\item Información de expertos
	\item Regresión simple
\end{itemize}

\section{Estudio técnico}

El estudio tecnico tiene 4 componentes basicas:

\subsection{La ingenieria del proyecto}

El alcance de la ingenieria del proyecto debe llegar a determinar la funcion de producción optima para la utilización eficiente
y eficaz de los recursos disponibles, para la producción del bien o servicio deseado. La adecuada combinación de la cantidad de
insumos con la cantidad que se desea obtener. Se buscan las cantidades que lleven a un costo óptimo sin afectar la calidad
del producto.

\subsubsection{Proceso de producción}

Es la elaboración del producto mediante una tecnología, un proceso que lleva la materia prima al producto 
final. Puede ser en {\bf serie, por pedido o por proyecto}.

\subsubsection{Efectos económicos de la ingeniería}

El proceso productivo y la tecnología seleccionada influirán directamente en la cuantía de las inversiones,
costos e ingresos del proyecto.

\subsubsection{Masa crítica técnica}

Deslandes plantéa que para medir la capacidad de competir, debe estimarse el costo fabril en distintos niveles de la
capacidad de producción, a partir de los componentes mas relevantes del costo. El costo fabril definido debe 
compararse con la capacidad de producción y el monto de la inversión, a esta relación se le denomina masa crítica
técnica.

	\[\frac{P_2}{P_1} = \frac{C_2}{C_1}^{-a} \]

Donde $P$ es el costo unitario de la operación, $C$ es la capacidad de la planta en unidades de producto, $a$ es el
factor de volumen.

	\[ \frac{Q_2}{Q_1} = \frac{C_2}{C_1}^{-b} \]

Donde $Q$ es el costo de equipos por unidad de capacidad y $b$ es un factor de volumen.

	\[ \frac{I_2}{I_1} = \frac{C_2}{C_1}^{f} \]

Donde $I$ es inversión total y $f$ es un factor de volumen. Un factor de volumen es la capacidad máxima de alguna 
variable que tiene una planta en un momento determinado.

\subsection{Elección entre alternativas tecnológicas}

Guadaquni, propuso que bajo un supuesto de ingresos iguales se debe seleccionar la alterniativa de menor valor
actualizado de sus costos. La alternativa de mayor riesgo es la de mayor intensidad de capital.
Dervisiotis propuso que para el mismo supuesto, es procedente que se calcule el costo de diferentes tecnologías
pero a distintos niveles de producción. Para elegir se siguen los pasos:

\begin{itemize}
	\item Se presenta el modelo opcional que incorpora todos los elementos que componen el costo total de cada una, y
	que el modelo tradicional no permite considerar. 
	\item Se incluye el concepto de rentabilidad para recuperar la inversión de capital y el efecto del costo 
	de capital (Financiación de capital fijo y de trabajo y retorno del inversionsta).
	\item Se incluyen además los efectos tributarios correspondientes. 
	\item La recuperación de la inversión se hacer de manera porrateada a lo largo de la vida útil del proyecto.
\end{itemize}

\subsection{Factores cualitativos}

Los factores no económicos mas comunes de considerar son: La disponibilidad de insumos, la oportunidad de su
abastecimiento (Material, humana o finaciera), La flexibilidad de la adaptación de la tecnología a distintas
condiciones de procesamiento de materias primas y la capacidad para expandir o contraer los niveles de producción
frente a estacionalidades en el proceso o frente a la inestabilidad al flujo de abastecimiento de materas primas.

\subsection{Factores cuantitativos}

	\[I_i=\frac{I_o-V_d}{n}\]
	\[K=K_f+K_v\]
	\[K=D*A\]
	\[K_v=K_{vd} + K_{va}\]
	\[C_i=Id + iK{vd}\]
	\[C_r=rA + rKva \]
	\[R=[P_x-V_x-F-D_{ip}-iD-iK_{vd}](1-t) + D_{ep} - rA - rK_{va} - I_i \]

Donde:

\begin{itemize}
	\item $I_i$ es el valor prorateado de la inversión en cada periodo
	\item $I_o$ es la inversión inicial
	\item $V_d$ son los valores de deshecho de la inversión
	\item $n$ es el periodo de prorateo
	\item $K$ es el capital total, $K_f$ es capital fijo y $K_v$ es capital variable
	\item $D$ es capital fijo financiado con deuda
	\item $A$ es capital fijo financiado con recursos propios
	\item $K_{vd}$ es la parte del capital variable financiado con deuda
	\item $K_{va}$ es la parte del capital variable financiado con recursos propios
	\item $i$ es la tasa de interes del prestamo
	\item $r$ es la tasa de interes del inversionista
	\item $C_i$ es el costo de financiarse con deuda
	\item $C_r$ es el costo de financiarse con capital propio
	\item $J$ es el porcentaje de los ingresos por ventas
	\item $P$ es el precio y $x$ es el numero de unidades
	\item Suponiendo $K_v=j(Px)$, $K_v=J(P_x)d+J(Px)a$, $C_i=iD+iJ(Px)D$ y
	$C_r=rA+rJ(Px)a$
\end{itemize}

\subsection{Lange}

Propone que existe una relación entre la inversión inicial $I_o$ y la capacidad productiva del
proyecto.

	\[min\{D=I_o(C)+nC\}\]
	\[\frac{dI_o}{dC}=-n\]

donde:

$D$ es el costo total del proyecto, $I_o(C)$ es el costo inicial  en función de $C$, y $C$ son
los costos de operación.

En el costo de los materiales se incluye tanto los que se utilizan directamente, como de manera
indirecta en el proyecto. Los indirectos son por ejemplo: la comida del perro guardian, la aseadora,
etc. 

\subsection{La valoración economica de las variables}

\subsubsection{Inversiones en obra física}

Se identifican como las que se hacer para: Compra de terreno, construcciones remodelaciones y otras obras 
complementarias relacionadas principalmente con el área para el funcionamiento del sistema del proyecto. Se 
pueden derivar del estudio organizacional y del estudio de mercado. Pueden ser para oficinas, salas de venta,
áreas de producción y presentación de productos.

\subsubsection{Inversiones en equipamiento}

La operación normal de la planta de la empresa generada a partir del proyecto. Incluye maquinarias, 
herramieentas, accesorios, vehículos, moviliario y equipo en general. Se considera como vida útil la
máxima utilización de la máquina en circunstancias en las que debería considerarse el periódo óptimo de reemplazo.

\subsubsection{Balance de personal}

Va en dirección que tiene la planta de producción o de prestación de servicios.

\subsection{Las decisiones de tamaño}

En terminos optimos el tamaño no deberia ser mayor que la demanda actual y esperada del mercado, ni la cantidad
demandada menor que el tamaño minimo economico del proyecto. El tamaño de un proyecto mide la relación de la
capacidad productiva durante un periodo considerado normal para las caracteristicas de cada proyecto en particular.
En razón de ello se debe tener en cuenta:

\begin{itemize}
	\item Capacidad teorica: es aquel volumen de producción que, con tecnicas optimas, permite operar al minimo costo
	      unitario.
	\item Capacidad maxima: es el volumen maximo de producción que se puede lograr, sometiendo los equipos a pleno
	      uso, independientemente de los costos de producción que genere.
	\item Capacidad normal: es aquella que, en las condiciones que se estiman, regiran durante la ejecución del proyecto
	      ya implementado. De tal forma que permitan operar a un minimo de costo unitario. Nota: la capacidad de la planta
	      se determina para los periodos de operación maxima.
\end{itemize}

Variables determinantes del tamaño del proyecto:

\begin{itemize}
	\item Dimensión del mercado.
	\item Tecnologia del proceso productivo.
	\item Disponibilidad de insumos.
	\item Localización.
	\item Financiamiento del proyecto.
\end{itemize}

\subsection{Las decisiones de localización}

La localización tiene un efecto condicionante en la tecnología utilizada. Por la variabilidad de los costos de operación. El objetivo
del estudio de localización es elegir aquella que produzca mayores ganancias entre la alternativas que se consideran factibles.

Factores que inciden en la localización:

\begin{itemize}
	\item Medios y costos de transporte.
	\item Disponibilidad y costos de la mano de obra.
	\item Cercania de las fuentes de abastecimiento.
	\item Factores ambientales.
	\item Cercania del mercado.
	\item Costo y disponibilidad de terrenos.
	\item Topografia de suelos.
	\item Estructura impositiva y legal.
	\item Disponibilidad de agua, energia, internet, y otros insumos.
	\item Comunicaciones.
	\item Posibilidad de desprenderse de deshechos.
	\item Tiempo asegurable para poder permanecer en el sitio escogido.
	\item Posibilidad de expansión padera nuevos proyectos o para incrementar el tamaño actual.
\end{itemize} 

El tamaño minimo es un punto de equilibrio entre los ingresos y los egresos, teniendo en cuenta que los ingresos deben incluir un 
porcentaje de utilidad. El tamaño maximo del proyecto esta limitado por el mercado y el capital.

\subsection{El proyecto como un proceso}

\begin{itemize}
	\item Estudio de mercado
	\item Estudio tecnico
	\item Estudio organizacional
	\item Estudio financiero
\end{itemize}

\section{Estudio organizacional}

El estudio organizacional tiene 3 componentes basicas:

\subsection{Costo de los aspectos organizacionales}

Dimensionamiento fisico de oficinas y su equipamiento, para calcular las inversiones en construcción y alojamiento, el nivel de los
cargos ejecutivos, para calcular el costo de las remuneraciones. El flujograma nos dice como funciona el organigrama, la ruta que se
tiene que seguir para cada proceso. El diagrama reparte tiempos y espacios.

La tendencia actual es que el diseño organizacional se haga de acuerdo con la dirección.

Las estructuras se referieren a las relaciones relativamente fijas existentes entre los puestos de una organización, y son el resultado
de los procesos de división del trabajo, departamentalización, esferas de control y delegación. El diseño de la estructura organizativa
requiere fundamentalmente de la definición de la naturaleza y contenido de cada puesto de la organización. Para poder estimar asi el
costo en remuneraciones administrativas del proyecto. Toda estructura debe tener un manual de funciones distinto al reglamento al que
se tiene que acoger la institución o la entidad.

El estudio de los aspectos organizacionales debe tener en cuenta que debe haber un estudio de la organización del proyecto basado en:

\begin{itemize}
	\item El principio de la división del trabajo para lograr la especialización.
	\item El principio de la unidad de dirección que postula la agrupación de actividades que tienen un objetivo comun bajo la dirección
	      de un solo administrador.
	\item El principio de la centralización, que establece el equilibrio entre centralización y descentralización.
	\item El principio de autoridad y responsabilidad.
\end{itemize}

\subsection{Costo de los sistemas y procedimientos administrativos}

\subsection{Costos de la operacion administrativa}

Depreciación de equipos de oficina.

\subsection{Marco legal dentro del estudio organizacional}

Como esta el proyecto dentro de la constitución politica de Colombia, que articulos de la constitución tienen que ver con el mismo. Cuales leyes
afectan el proyecto, cuales codigos tienen que ver, cuales decretos leyes lo afectan, decretos presidenciales, resoluciones, memorandums, circulares,
directrices, ordananzas, decretos departamentales, acuerdos, decretos municipales.

\section{Estudio financiero}

El estudio financiero contiene:

\begin{itemize}
	\item Flujo de caja proyectado.
	\item Evaluacion financiera.
	\item Analisis de riesgo.
\end{itemize}

El estudio financiero tiene como proposito:

\begin{itemize}
	\item Ordenar y sistematizar la información de caracter monetario obtenida en las etapas anteriores.
	\item Elaborar los cuadros analiticos y antecedentes adicionales para la evaluación del proyecto.
\end{itemize}

En un sistema de información financiera, en una etapa previa se organiza la información de caracter monetario, ya sea como inversiones, costos o 
ingresos.

Se hacen las siguientes clasificaciones:

\subsection{Inversiones}

\begin{itemize}
	\item Capital de trabajo.
	\item Terrenos.
	\item Obras fisicas.
	\item Equipamiento de fabrica / oficina.
	\item Recursos para ampliación, futuro crecimiento y renovaciones.
\end{itemize}

\subsection{Activos}

\begin{itemize}
	\item Activos fijos.
	\item Activos nominales.
	\item Capital de trabajo.
\end{itemize}

\subsection{Costos}

\begin{itemize}
	\item Estudio tecnico: costos asociados a producir un bien o prestar un servicio.
	\item Estudio organizacional: funcionamiento y administración.
	\item Impuestos a las utilidades.
\end{itemize}

\subsection{Fuentes}

\begin{itemize}
	\item Recursos propios.
	\item Credito bancario.
	\item Credito de proveedores.
	\item Posibles inversionistas.
	\item Aportes (subsidios) del estado.
\end{itemize}

\subsection{Condiciones de pago}

\begin{itemize}
	\item Efectivo.
	\item Titulos valores.
	\item Pagos de terceros.
\end{itemize}

\subsection{Capital de trabajo}

Activos corrientes que se requieren para el funcionamiento normal del proyecto durante un ciclo productivo. Un ciclo productivo es desde que empieza
a planear el producto hasta que lo vende.

\subsection{Capital de trabajo neto}

Capital de trabajo = activo corriente \newline
Capital de trabajo neto = activo corriente - pasivo corriente \newline
Activo corriente = activo liquido + activo realizable \newline
Activo liquido = caja + bancos + efectivo \newline
Activo realizable = inventario + titulos valores \newline
Pasivo corriente = salarios + servicios + proveedores + impuestos + arriendos + deudas laborales ya causadas

\subsection{Tasa de costo de capital}

Representa una medida de la rentabilidad minima que se exigira en el proyecto según su riesgo, de manera tal que el retorno esperado permita cubrir: 

\begin{itemize}
	\item La totalidad de la inversión inicial.
	\item Los egresos de la operación.
	\item Los intereses que deberan pagarse por aquella parte de la inversión financiada con prestamos.
	\item La rentabilidad que el inversionista le exige a su propio capital invertido en el proyecto. 
\end{itemize}

Los ingresos del proyecto deben estar por encima de la tasa de costo de capital, de lo contrario el proyecto generaria perdidas.  

\subsection{Costo de utilizar recursos de deuda}

Kd: Costo de la deuda de un proyecto antes de los impuestos.
Kd(1 - t): Deuda de un proyecto despues de pagar los impuestos.
t: Tasa marginal de impuestos

\subsubsection{Ejemplo}

Suponga que un proyecto tiene las siguientes cifras:

Utilidad antes de intereses e impuestos -> 10.000 dolares anuales
Inversión -> 40.000 dolares
Tasa de interes del prestamo -> 11\% anuales
Tasa de impuestos -> 40\%

Se requiere calcular su financiamiento con recursos de la deuda y con capital propio.

\begin{table}
\begin{tabular}{ |c|c|c| }
\hline
                                &   Con deuda          &           Sin deuda \\
\hline
Utilidad antes de int e imp     &  \$10.000            &           \$10.000\\
Intereses		 	&   \$4.400            &                \$0\\
Utilidad antes de impuestos     &   \$5.600            &           \$10.000\\
Impuestos			&   \$2.240            &            \$4.000\\
Utilidad neta			&   \$3.360            &            \$6.000\\
\hline
\end{tabular}
\end{table}

CRD = \$6.000 - \$3.360 = \$2.640
CPD = (CRD / I) * 100\% = 6.6\%

Kd = 11\%
Kd(1 - t) = 0.11 * (1 - 0.4) = 6.6\%
Kd(1 - t) = CPD

\subsection{Coso de utilizar capital propio o patrimonial}

El capital patrimonial de un proyecto es aquella parte de la inversión que se debe financiar con recursos propios. Se toma como costo de capital propio
la tasa Kp, asociada con la oportunidad de inversión de riesgo similar que se abandonara por destinar esos recursos al proyecto objeto de analisis.

\subsubsection{Costo ponderado de capital}
\[ K_0 = K_d\frac{D}{V}+K_p\frac{p}{v} \]
\[ K'_0 = K_d(1-t)\frac{D}{V}+K_p\frac{p}{v} \]
Donde $K_0$ es la tasa ponderada de capital, $K_p$ es la tasa con recursos propios, $K_d$ es la tasa del préstamo, $D$ es el monto de la deuda, $P$ es el monto del préstamo, $V$ es el valor de la firma en el mercado, $K'_0$ es la tasa ponderada de capital corregida por efectos tributarios y $t$ es la tasa marginal de impuestos.

\subsection{Valor agregado}
Si el valor agregado neto (VAN) es positivo representa el excedente obtenido por el inversionista luego de la inversión, los gastos financieros y la rentabilidad exigida por el inversionista.
Si al flujo del proyecto se le descuentan los intereses y amortizaciones el saldo equivaldría a la recuperación del aporte del inversionista mas la fanancia por él exigida.y un excedente igual al VAN del proyecto.
\begin{table}
 \begin{tabular}{|c|c|c|c|c|}
  \hline
   & 0 & 1 & 2 & 3 \\
  \hline
  Flujo & (100000) & 39590 & 39590 & 39590 \\
  \hline
 \end{tabular}
  \caption{Tasa de descuento del inversionista}
\end{table}

Si la inversión se financia en un 60\% con deuda al 8\% y el resto con recursos propios al 12\%. Y si la tasa de impustos para la empresa es del 10\% sobre las utilidades se requiere calcular la tasa ponderada de capital corregida por efectos tributarios, y calcular la tasa con que se obtendría un VAN igual a cero.

La tasa interna de retorno es aquella con la cual el valor agregado neto es igual a 0.


\end{document}