
\documentclass{beamer}
\mode<presentation>{ \usetheme{boxes} }

\usepackage[utf8]{inputenc}
\usepackage[spanish]{babel}
\usepackage{amsmath, amsthm, amsfonts}
\addto\captionsspanish{\def\refname{\textsc{Bibliografía}}}

\title {Propuesta Proyecto de Grado}
\author { Santiago Gutierrez - Sebastián Gómez - Jorge Adrián Martinez }
\date {Agosto de 2010}

\usetheme{Warsaw}
\begin{document}
	
	\frame{\titlepage}
	
	\frame{\tableofcontents}
	
	\section{Propuesta de proyecto de grado}
	\subsection{Introducción}
	\begin{frame}
		\frametitle{Título provisional}
		Desarrollo de un sistema de reconocimiento óptico de caracteres para
		celulares, que funcione bajo condiciones controladas.
	\end{frame}
	
	\subsection{Breve descripción general del problema}
	\begin{frame}
	\frametitle{Población Objetivo}
	\begin{itemize}
		\pause
		\item Según el DANE en su gran censo nacional realizado en el año 2005, se estima que el 6.3\% de la población colombiana sufre de alguna discapacidad. De estas personas con discapacidad se estima que el 43.4\% tienen dificultades para ver aun con lentes.
		\pause
		\item La organización mundial de la salud estima que en el mundo hay 161 millones de personas con problemas visuales, de los cuales 37 millones son ciegos.
	\end{itemize}
	\end{frame}
	
	\begin{frame}
	\frametitle{Problemas técnicos I}
	\begin{itemize}
		\pause
		\item Después de una búsqueda en más de 20 artículos científicos relacionados al tema de reconocimiento óptico de caracteres, y de revisar la documentación técnica de varios OCRs libres, no se encontró un sistema de reconocimiento de caracteres por un medio óptico que haya sido desarrollado para teléfonos inteligentes y que sea de código libre y abierto.
		\pause
		\item En los computadores de escritorio las imágenes se obtienen por medio de un escáner, logrando condiciones de iluminación muy buenas y uniformes. En los teléfonos inteligentes las imágenes se obtienen por medio de una cámara fotográfica, haciendo más relevantes diversas variables, como por ejemplo: la inclinación del texto, iluminación no uniforme, deformaciones elásticas del texto debido a la forma del papel en reposo.
	\end{itemize}
	\end{frame}
	
	\begin{frame}
	\frametitle{Problemas técnicos II}
	\begin{itemize}
		\pause
		\item La memoria y la capacidad de procesamiento son muy reducidas en los teléfonos inteligentes con respecto a los computadores de escritorio.
		\pause
		\item El tamaño de la memoria cache es mucho más pequeño en los teléfonos inteligentes. Como resultado de esto, los algoritmos diseñados para computadores de escritorio pueden ser muy lentos en los teléfonos inteligentes, pues se hicieron pensando en caches de varios Mbytes.
		\pause
		\item Un computador de escritorio normalmente tiene instaladas múltiples librerías de propósito general que son utilizadas por distintos programas, un teléfono inteligente, en cambio, solo provee las librerías más básicas.
	\end{itemize}
	\end{frame}
	
	\subsection{Justificación Inicial o Preliminar}
	\begin{frame}
	\frametitle{Beneficios para personas invidentes}
	\begin{itemize}
	\item Un menor costo, al no ser necesario adquirir un computador de escritorio, escáner o impresora braille.
	\pause
	\item Portabilidad: ninguno de los sistemas anteriormente mencionados esta diseñado para ser llevado en todo momento por el usuario, dificultando el acceso al conocimiento e información que no se encuentre en el lugar físico en el que se encuentra alguno de estos sistemas.
	\pause
	\item Una solución económica que pueda ejecutarse en un dispositivo móvil, sin necesidad de hardware adicional, haría posible que cada invidente pudiera tener su propio sistema de OCR personal, mientras que los otros sistemas usualmente (por su costo) son adquiridos por instituciones en cantidad limitada.

	\end{itemize}
	\end{frame}
	
	\begin{frame}
	\frametitle{Otros beneficios}
	\begin{itemize}
	\item Aportar al conocimiento: Al desarrollar este aplicativo se tendrán que hacer pruebas experiméntales cuyos resultados aporten a otras investigaciones.
	\end{itemize}
	\end{frame}
	
	\subsection{Objetivo provisional}
	\begin{frame}
	\frametitle{Objetivo provisional}
	Desarrollar un sistema prototipo de reconocimiento óptico de caracteres para celulares inteligentes, usando algún conjunto de algoritmos específicos escogidos con base en resultados de investigaciones previas reconocidas.
	\end{frame}
	
	\subsection{Clase de investigación}
	\begin{frame}
	\frametitle{Clase de investigación}
	En esta investigación se utilizará un enfoque cuantitativo.
	\end{frame}
	
	\subsection{Recursos disponibles}
	\begin{frame}
	\frametitle{Recursos disponibles}
	\begin{itemize}
		\item Computador portátil Dell Inspiron 1420\\
			Intel Centrino Duo, 2GB de memoria RAM
		\item Celular Nexus One, con sistema operativo Android\\
			Procesador ARM, 512 MB de memoria RAM, Cámara 5 MPixeles
		\item Base de datos de artículos publicados en IEEE explore
		\item Código fuente del OCR experimental de licencia libre Ocropus
	    \item Recursos bibliográficos de la biblioteca de la Universidad Tecnológica de Pereira y propios en el área de inteligencia artificial, reconocimiento óptico de caracteres y procesamiento digital de imágenes.
	\end{itemize}
	\end{frame}
	
\bibliographystyle{plain}
\bibliography{refs}
\nocite{*}
\end{document}
