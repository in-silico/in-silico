
\documentclass{article}

\begin{document}
\title {Taller 2 - Proyecto de grado I}
\author { Santiago Gutierrez - Sebastián G\'omez }
\date {Agosto de 2010}
\maketitle

	\section{Introducci\'on}
	El problema de reconocimiento de caracteres no es un problema f\'acil, pero algunos
	avances significativos se han hecho en la materia; Adem\'as de ser un tema que puede
	contribuir al mejoramiento de la calidad de vida a las personas invidentes al darles
	una forma de acceder a textos impresos.
	En este taller pretendemos mostrar algunos de los avances cient\'ificos en el \'area
	de reconocimiento de caracteres a nivel mundial, as\'i como mostrar los estudios
	realizados en la Universidad Tecnol\'ogica sobre temas relacionados con la asistencia
	a personas discapacitadas o reconocimiento de patrones.
	
	\section{Reconocimiento de caracteres y analisis de documentos}
	En esta secci\'on hablaremos de algunas investigaci\'ones realizadas en el \'area
	de reconocimiento de caracteres y de an\'alisis de documentos.
	Las etapas necesarias para llevar a cabo el proceso de an\'alisis de documentos
	y reconocimiento de caracteres son: Binarizaci\'on de la im\'agen
	
	\section{Sistemas de asistencia a personas invidentes}

\end{document}
