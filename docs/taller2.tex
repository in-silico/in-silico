
\documentclass{article}

\usepackage[utf8]{inputenc}
\usepackage[spanish]{babel}

\begin{document}
\title {Taller 2 - Proyecto de grado I}
\author { Santiago Gutierrez - Sebastián Gómez }
\date {Agosto de 2010}
\maketitle

	\section{Introducción}
	El problema de reconocimiento de caracteres no es un problema fácil, pero algunos
	avances significativos se han hecho en la materia; Además de ser un tema que puede
	contribuir al mejoramiento de la calidad de vida a las personas invidentes al darles
	una forma de acceder a textos impresos.
	En este taller pretendemos mostrar algunos de los avances científicos en el área
	de reconocimiento de caracteres a nivel mundial, así como mostrar las tesis
	realizados en la Universidad Tecnológica sobre temas relacionados con la asistencia
	a personas discapacitadas o reconocimiento de patrones.
	
	\section{Reconocimiento de caracteres y análisis de documentos}
	En esta sección hablaremos de algunas investigaciónes realizadas en el área
	de reconocimiento de caracteres y de análisis de documentos.
	Las etapas necesarias para llevar a cabo el proceso de análisis de documentos
	y reconocimiento de caracteres son: Binarización de la imágen, análisis y
	segmentación del documento y reconocimiento óptico de caracteres (OCR).
	
	\subsection{Binarización de imágen}
	Esto se refiere a tomar una imágen en escala de grises y convertirla a una 
	imágen en blanco y negro. Pero se puede ver como separar la parte que nos interesa
	(foreground) de la parte que no nos interesa (background).
	Una técnica que da buen resultado es la binarización de Sauvola \cite{ocropus2},
	que realiza una binarización basada análisis local de la imágen. El hecho de
	ser local brinda una mejor binarización en condiciones de luz variable.
	
	\subsection{Análisis del documento}
	Segun \cite{doc_analysis2}, esta etapa puede ser vista como un análisis sintáctico,
	en el que el documento se divide en sus partes estructurales como un árbol de análisis
	sintáctico. Como todo análisis gramatical, se puede seguir los enfoques
	\textit{buttom-up} y \textit{top-down}. Un enfoque \textit{buttom-up} comenzaría desde
	los pixeles y luego los relacionaría en grupos mas grandes (Caracteres, Palabras, Lineas
	o Parrafos). Un enfoque \textit{top-down} por el contrario, comienza con la imágen de un
	documento y comenzaría a dividirla hasta llegar a sus componentes.
	Cada algoritmo tiene limitaciones, ventajas y distintos requerimientos de máquina.
	Una investigación realizada sobre diferentes algoritmos \cite{benchmark1}, los
	mejores resultados sobre diferentes documentos, con distintas estructuras físicas,
	fueron para Docstrum \cite{docstrum93} y para Voronoi \cite{voronoi1}, con procentajes
	de error del 4.3\% y 4.7\% respectivamente. Otro método
	llamativo es el \textit{RLSA}, que llama la atención por su sencillez y puede bindar
	buenos resultados en documentos \textit{Manhattan-Layout}\cite{RLSA1}.
	Probablemente la mejor solución para nuestra aplicación pueda resultar de la mezcla de
	varios de los algoritmos existentes, en \cite{voronoi2} se como los se pueden mejorar
	los resultados mezclando los beneficios de diferentes algoritmos para soluciones
	particulares.
	
	\subsection{Reconocimiento óptico de caracteres OCR}
	Una vez se tienen separados los caracteres en el documento, se debe reconocer a que
	caracter corresponde el grupo de pixeles que entregó la etapa del análisis del
	documento.
	
	\section{Sistemas de asistencia a personas invidentes}
	
\bibliographystyle{plain}
\bibliography{refs}
\end{document}
