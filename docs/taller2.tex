
\documentclass{article}

\usepackage[utf8]{inputenc}
\usepackage[spanish]{babel}

\begin{document}
\title {Taller 2 - Proyecto de grado I}
\author { Santiago Gutierrez - Sebastián Gómez }
\date {Agosto de 2010}
\maketitle

	\section{Introducción}
	El problema de reconocimiento de caracteres no es un problema fácil, pero algunos
	avances significativos se han hecho en la materia; Además de ser un tema que puede
	contribuir al mejoramiento de la calidad de vida a las personas invidentes al darles
	una forma de acceder a textos impresos.\newline
	En este taller pretendemos mostrar algunos de los avances científicos en el área
	de reconocimiento de caracteres a nivel mundial, así como mostrar las tesis
	realizados en la Universidad Tecnológica sobre temas relacionados con la asistencia
	a personas discapacitadas o reconocimiento de patrones. Cabe anotar que los 3 puntos
	propuestos en el taller están resueltos pero de forma dispersa, dándole continuidad
	al texto para mostrar como las referencias encontradas nos pueden ayudar en el
	problema particular a tratar (punto 2 del taller), expresando los problemas o
	limitaciones de algunas de las soluciones planteadas en los artículos (Punto 1) y
	comentando algunos de los proyectos de grado previamente realizados en temáticas
	relacionadas con el problema del reconocimiento de patrones o asistencia a las
	personas invidentes (Punto 3).
	
	\section{Reconocimiento de caracteres y análisis de documentos}
	En esta sección hablaremos de algunas investigaciones realizadas en el área
	de reconocimiento de caracteres y de análisis de documentos.
	Las etapas necesarias para llevar a cabo el proceso de análisis de documentos
	y reconocimiento de caracteres son: Binarización de la imagen, análisis y
	segmentación del documento y reconocimiento óptico de caracteres (OCR).
	
	\subsection{Binarización de imagen}
	Esto se refiere a tomar una imagen en escala de grises y convertirla a una 
	imagen en blanco y negro. Pero se puede ver como separar la parte que nos interesa
	(foreground) de la parte que no nos interesa (background).
	Una técnica que da buen resultado es la binarización de Sauvola \cite{ocropus2},
	que realiza una binarización basada análisis local de la imagen. El hecho de
	ser local brinda una mejor binarización en condiciones de luz variable.
	
	\subsection{Análisis del documento}
	Según \cite{doc_analysis2}, esta etapa puede ser vista como un análisis sintáctico,
	en el que el documento se divide en sus partes estructurales como un árbol de análisis
	sintáctico. Como todo análisis gramatical, se puede seguir los enfoques
	\textit{buttom-up} y \textit{top-down}. Un enfoque \textit{buttom-up} comenzaría desde
	los píxeles y luego los relacionaría en grupos mas grandes (Caracteres, Palabras, Lineas
	o Párrafos). Un enfoque \textit{top-down} por el contrario, comienza con la imagen de un
	documento y comenzaría a dividirla hasta llegar a sus componentes.
	Cada algoritmo tiene limitaciones, ventajas y distintos requerimientos de máquina.
	Una investigación realizada sobre diferentes algoritmos \cite{benchmark1}, indica que los
	mejores resultados sobre diferentes documentos, con distintas estructuras físicas,
	fueron para Docstrum \cite{docstrum93} y para Voronoi \cite{voronoi1}, con porcentajes
	de error del 4.3\% y 4.7\% respectivamente. Otro método
	llamativo es el \textit{RLSA}, que llama la atención por su sencillez y puede brindar
	buenos resultados en documentos \textit{Manhattan-Layout}\cite{RLSA1}.
	Probablemente la mejor solución para nuestra aplicación pueda resultar de la mezcla de
	varios de los algoritmos existentes, en \cite{voronoi2} se como los se pueden mejorar
	los resultados mezclando los beneficios de diferentes algoritmos para soluciones
	particulares.
	
	\subsection{Reconocimiento óptico de caracteres OCR}
	Esta etapa se puede clasificar como un subproblema de reconocimiento de patrones.
	
	Existen varias aproximaciones para solucionar el problema de reconocimiento de patrones; 
	modelos inspirados en la biología como los que usan redes neuronales\cite{im_biology}. Otros
	basados en modelos matemáticos como los momentos invariables de hu\cite{art_hu}, o los
	descriptores de Fourier. El principio fundamental de esta clase de algoritmos son
	los descriptores invariantes, estos son: funciones para evaluar imágenes, que retornan valores 
	parecidos o idénticos cuando se aplican	transformaciones afines a estas, como la rotación,
	el cambio de escala, la traslación entre otras \cite{ocrs1}.
	
	Generalmente lo primero que se hace es tomar muchas imágenes de un solo tipo: sea una letra, un 
	numero o una forma. medir sus características aplicando
	descriptores invariantes, y manualmente etiquetar esta imagen,(una a, un 1, un rombo etc) 
	guardando esta información en una base de datos. 
	
	Luego cuando el sistema se pone en marcha se obtienen las características de la imagen, y se
	aplica un algoritmo de aprendizaje sobre dichas características. Uno de los métodos mas
	sencillos consiste en tomar la distancia euclidiana normalizada entre valores de la misma
	clase y en clases diferentes. Con esto se pueden crear regiones en el hiper-espacio (Un
	espacio de varias dimensiones) para las diferentes clases, y así cuando llegue una imagen
	nueva a clasificar solo se deben tomar las características y ubicar en que que región del
	hiper-espacio se encuentra este objeto \cite{learn1}.
		
	
	\section{Sistemas de asistencia a personas invidentes}
	La tecnología actual ha permitido a las personas invidentes tener acceso a información
	y conocimiento al que antes no tenían acceso; Entre estas tecnologías se encuentran los
	lectores de pantalla, impresoras y sistemas refrescables de Braille, entre otros.
	Un proyecto interesante desarrollado en la Universidad Tecnológica de Pereira, permite
	a los invidentes reconocer imágenes con colores a través de vibraciones a diferentes
	frecuencias\cite{iris04}; Para ello, usan un guante con imanes permanentes y una matriz
	de electroimanes conectada al computador por el puerto USB.
	En el área de acceso al texto, existen sistemas que pasan el texto en ASCII a voz
	conocidos como sistemas \textit{Text-To-Speech}. Y también sistemas para el computador
	que permiten pasar texto en imagen a texto en ASCII, conocidos como OCR; Estos funcionan
	usualmente tomando una imagen de un escáner. Según pruebas de eficacia realizadas por
	la UNLV, los OCRs libres con mejores resultados son	\textit{Tesseract} y el 
	\textit{Ocropus}, cuyo funcionamiento general se explica en \cite{tesseract1} y
	\cite{ocropus1} respectivamente.\newline
	Cuando el problema cambia de tomar la imagen con un escáner a tomarla con una cámara,
	se deben seguir diferentes enfoques, \cite{doc_analysis3} halló los retos mas
	grandes son la proyección de la imagen en perspectiva en el plano y la resolución
	requerida para poder reconocer bien los caracteres.
	En cuanto a hacer que el OCR sea realizado por un celular, encontramos investigaciones
	como \cite{mob_smallpic}, en la que se trabajaban con imágenes de muy baja resolución
	(640x480) y como resultado solo pueden reconocer textos muy cortos como avisos y carteles
	y no documentos completos.
	En \cite{mob_withpc}, para tratar de equilibrar el hecho de tener una cámara de baja
	resolución, toman varias imágenes de baja resolución en lugar de una sola, uniéndolas
	luego para lograr una mejor resolución. Esto sin embargo conlleva a un incremento la
	carga de procesamiento, haciendo que no sea posible el procesamiento dentro del celular.
	Así que restringieron su solución para que usando Internet se enviaran las imágenes a un
	servidor, que fuera este quien las procesara y enviara los resultados.\newline
	Cabe anotar además que ninguna de las tecnologías anteriormente mencionadas para hacer
	el OCR en el celular, fue diseñada para ser usada por personas invidentes. Ya que el 
	primero no fue diseñado para reconocer documentos, y el segundo requeriría que la persona
	invidente supiera con anterioridad como está la estructura del documento para poder mover
	la cámara del celular en el orden en el que está el texto en el documento.
	
\bibliographystyle{plain}
\bibliography{refs}
\end{document}
