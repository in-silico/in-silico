
\documentclass{article}

\usepackage{mathtools}

\begin{document}
\title {Shape recognition using machine learning}
\author {Sebastian Gomez}
\date {July of 2010}
\maketitle

	\section{Introduction}
	When you teach a child how to recognize geometrical figures, such as
	a triangle, a square or a rectangle; You don't talk about parallel sides of 90
	degrees angles. The only thing that you usually do is to show him examples
	of those figures and the child learn by his own how to differentiate 
	them. One way of making something similar in a computer, is by using
	machine learning; Even though to make a computer able to differentiate 
	figures is not a simple task, we are going to illustrate some simple steps
	to achieve that in this article. The field of computer vision usually
	propose 3 stages for image recognition.
	
	\section{Image filtering}
	The first stage receives as input an image an outputs an image having only
	what we need. In order to recognize geometrical figures, we must find the 
	contour of the figure. But before doing so we turn the image into gray scale.
	As any signal acquisition, the input has noise and we must filter it; To do
	so we used a Gaussian filter. We could define now the filtered grayscale image
	as a function $I_1(x,y)$, that represents the value of the pixel located at
	$(x,y)$ as an integer between 0 and 255; Where 0 is black, 255 is white and
	all other values are some kind of gray.\newline
	As we are interested in finding the contour of the figure, we first binarize
	the image. That binarization is achieved by choosing a value T, such that any
	value above it is taken to 255, and any value below it is taken to 0.
	To choose T as a constant value is not a good idea, since it will make the
	system too dependent on light conditions. We used an approach called
	adaptative threshold, that choose the value of T for each pixel by averaging
	its neighbors. If $I_2(x,y)$ represents the binarized image and $\mu(x,y)$
	represents the mean in some area $A_2$ arround $(x,y)$; We would have:
	\[ I_2(x,y) = \left\{ \begin{array}{ll}
		0   & \mbox{if $I_1(x,y) < u(x,y)$} \\
		255 & \mbox{otherwise}
	\end{array} \right. \]
	
	\section{Obtaining the features}
	The second stage receives as input $I_2(x,y)$, the output image of the first
	stage; and should return to us the features for the recognition stage. We used the 
	OpenCV library for all the project, this library has an utility to find the
	contour of a figure once the image is binarized. The contour is a sequence of
	(x,y) points at the border of an image, that we are going to approximate with a
	polygon using an algorithm of the library. That algorithm receives a sequence
	of points, take the farthest points of that sequence as part of the polygon;
	Then it find the farthest point from the lines of the polygon and 

\end{document}
