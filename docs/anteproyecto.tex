\documentclass[a4paper, 11pt, oneside]{article}


% idioma
\usepackage[utf8]{inputenc}
\usepackage[spanish]{babel}

%tablas
\usepackage{booktabs}

%rotar tablas
\usepackage{rotating}

%color tablas
\usepackage{colortbl}

%espaciado
\usepackage{setspace}
\onehalfspacing
\setlength{\parindent}{0pt}
\setlength{\parskip}{2.0ex plus0.5ex minus0.2ex}


%margenes según n. icontec
\usepackage{vmargin}
\setmarginsrb           { 4.0cm}  % left margin
                        { 3.0cm}  % top margcm
                        { 2.0cm}  % right margcm
                        { 3.0cm}  % bottom margcm
                        {  10pt}  % head height
                        {0.25cm}  % head sep
                        {   9pt}  % foot height
                        { 0.3cm}  % foot sep

% inserción url's notas de pie.
\usepackage{url}


% Paquetes de la AMS:
\usepackage{amsmath, amsthm, amsfonts}
\addto\captionsspanish{\def\refname{\textsc{Bibliografía}}}

\newcommand\portada{
	\begin{titlepage}
		\begin{center}
			{\large \bf DESARROLLO DE UN SISTEMA DE RECONOCIMIENTO ÓPTICO DE CARACTERES PARA CELULARES, QUE FUNCIONE BAJO CONDICIONES CONTROLADAS }
			\vfill
% 			{\large\bf PRESENTADO POR \par}
			{\large\bf SEBASTIÁN GÓMEZ GONZÁLEZ \par}
			{\large\bf SANTIAGO GUTIERREZ ALZATE \par}
			\vfill
			{\large\bf UNIVERSIDAD TECNOLÓGICA DE PEREIRA  \par}
			{\large\bf FACULTAD DE INGENIERÍAS \par}
			{\large\bf INGENIERÍA DE SISTEMAS Y COMPUTACIÓN \par}
			{\large\bf PEREIRA\par}
			{\large\bf SEPTIEMBRE DE 2010 \par}
		\end{center}
	\end{titlepage}
}


\begin{document}
\portada
%\contraportada

	\clearpage
	\section{Título}
	Desarrollo de un sistema de reconocimiento óptico de caracteres para celulares, que funcione bajo condiciones controladas.
	
	\section{Formulación del problema}
	Los problemas visuales afectan a una gran parte de la población, tanto en Colombia\footnote{Según el DANE en su gran censo nacional realizado en el año 2005, se estima que el 6.3\% de la población colombiana sufre de alguna discapacidad. De estas personas con discapacidad se estima que el 43.4\% tienen dificultades para ver aun con lentes.} como a nivel mundial\footnote{La organización mundial de la salud estima que en el mundo hay 161 millones de personas con problemas visuales, de los cuales 37 millones son ciegos.}, estos traen consigo consecuencias devastadoras para las personas que los sufren, afectando su calidad de vida significativamente. Entre todos los problemas que sufren las personas con discapacidades visuales, uno de los más críticos es el del acceso a la información; esto debido a que la mayoría de la información que recibimos del entorno llega a través de nuestros ojos. Los problemas para obtener información causan que las personas con discapacidades visuales se queden rezagadas con respecto a los demás miembros de la sociedad, impidiendo así que tengan las mismas oportunidades de vida que una persona con sus cinco sentidos intactos. Generalmente, esto causa un efecto de exclusión y de segregación en estas personas debido a que son vistas como una carga, tanto por si mismas como por las personas que los rodean. 
	
	Entre las soluciones que existen en el mercado para este problema, no encontramos un software económico para que las personas invidentes puedan tener acceso rápido a documentos escritos. Esto es debido a que el software tradicionalmente construido para reconocer caracteres ópticamente no esta diseñado para ser usado en dispositivos móviles. Después de una búsqueda en más de 20 artículos científicos relacionados al tema de reconocimiento óptico de caracteres, y de revisar la documentación técnica de varios OCRs libres, no encontramos un sistema de reconocimiento de caracteres por un medio óptico que haya sido desarrollado para teléfonos inteligentes y que sea de código libre y abierto.

	Existen grandes diferencias entre un computador de escritorio y un teléfono inteligente, entre ellas podemos encontrar:
	\begin{itemize}
	\item La memoria y la capacidad de procesamiento son muy reducidas en los teléfonos inteligentes con respecto a los computadores de escritorio.

	\item En los computadores de escritorio las imágenes se obtienen por medio de un escáner, logrando condiciones de iluminación muy buenas y uniformes. En los teléfonos inteligentes las imágenes se obtienen por medio de una cámara fotográfica, haciendo más relevantes diversas variables, como por ejemplo: la inclinación del texto, iluminación no uniforme, deformaciones elásticas del texto debido a la forma del papel en reposo.

	\item El tamaño de la memoria cache es mucho más pequeño en los teléfonos inteligentes. Como resultado de esto, los algoritmos diseñados para computadores de escritorio pueden ser muy lentos en los teléfonos inteligentes, pues se hicieron pensando en caches de varios Mbytes.

	\item Un computador de escritorio normalmente tiene instaladas múltiples librerías de propósito general que son utilizadas por distintos programas, un teléfono inteligente, en cambio, solo provee las librerías más básicas.
	\end{itemize}

	Así pues, con el fin de aportar a la solución de este problema, se hace necesario un sistema de reconocimiento óptico de caracteres diseñado específicamente para teléfonos inteligentes, ya que los sistemas que existen actualmente no fueron diseñados teniendo en cuenta estos factores. 
\clearpage


	\section{Justificación}
	A partir de la incidencia del problema, una forma de ayudar a que las personas con discapacidades visuales tengan una mejor calidad de vida es lograr que puedan tener acceso a la información que se encuentra en documentos escritos, tales como cartas, libros, y volantes. Un sistema que solucionara este problema, le posibilitaría a las personas con discapacidades visuales mejorar enormemente su calidad de vida, ya que les permitiría una mejor educación, al poder acceder a los mismos libros de texto que el resto de la sociedad, y no solo, como actualmente sucede, a los pocos libros que se encuentran disponibles en formato braille.

	Para dar una solución efectiva al problema se requiere de algún sistema que le permita a los discapacitados visuales acceder a la información escrita de una manera rápida y fácil, lo cual implica gran movilidad. Si bien existen sistemas de reconocimiento óptico de caracteres para computadores de escritorio, estos no poseen dicha movilidad, porque además del computador, también necesitan de un escáner para funcionar. Así pues, para que un sistema de este tipo pueda ser utilizado en múltiples lugares, se requiere de algún dispositivo que sea: pequeño, fácilmente transportable y lo suficientemente poderoso para hacer el trabajo en un tiempo razonable. Los celulares comunes cumplen las primeras dos características, pero no la tercera, ya que su poder de procesamiento es varias de veces inferior al de un computador común. Sin embargo, una nueva generación de celulares, que se denomina la de los \textit{celulares inteligentes}, cumple esta tercer característica, ya que su poder de computo es comparable con el de un computador portátil. Así pues, si existiese un sistema de reconocimiento óptico de caracteres diseñado para estos celulares, este podría utilizarse en un sistema que permitiese reconocer texto de imágenes adquiridas por medio de la cámara de estos celulares, que lograría mejorar la capacidad de acceso a la información de las personas con discapacidades visuales.

	Además de esto, un sistema móvil para reconocer teto de documentos impresos presentaría los siguientes beneficios respecto de los sistemas tradicionales como las impresoras braille y los lectores basados en escáner:
	\begin{itemize}
	\item Un menor costo, al no ser necesario adquirir un computador de escritorio, escáner o impresora braille.

	\item Portabilidad: ninguno de los sistemas anteriormente mencionados esta diseñado para ser llevado en todo momento por el usuario, dificultando el acceso al conocimiento e información que no se encuentre en el lugar físico en el que se encuentra alguno de estos sistemas.
	
	\item Una solución económica que pueda ejecutarse en un dispositivo móvil, sin necesidad de hardware adicional, haría posible que cada invidente pudiera tener su propio sistema de OCR personal, mientras que los otros sistemas usualmente (por su costo) son adquiridos por instituciones en cantidad limitada.

	\item Aportar al conocimiento: Al desarrollar este aplicativo se tendrán que hacer pruebas experiméntales cuyos resultados aporten a otras investigaciones.
	\end{itemize}
	\clearpage

	\section{Objetivo}
	\subsection{Objetivo general}
	Desarrollar un sistema prototipo de reconocimiento óptico de caracteres para celulares inteligentes, usando algún conjunto de algoritmos específicos escogidos con base en resultados de investigaciones previas reconocidas.
	\subsection{Objetivos específicos}
	%aqui van los específicos
	\clearpage
	
	\section{Marco referencial}
		
	\section{Diseño metodológico preliminar}
	
	\section{Personas que participan en el proyecto}
	\begin{itemize}
   \item Santiago Gutierrez\\
Estudiante Ingeniería de Sistemas y Computación\\
Universidad Tecnológica de Pereira\\
Santigutierrez1@hotmail.com
   \item Sebastián Gómez\\
Estudiante Ingeniería de Sistemas y Computación\\
Universidad Tecnológica de Pereira\\
sebasutp@gmail.com
   \item Jorge Adrián Martinez\\
Estudiante Ingeniería de Sistemas y Computación\\
Universidad Tecnológica de Pereira\\
nogardark@hotmail.com
   \item Saulo de Jesús Torres\\
Doctorado en Informática: Ingeniería de Software (Universidad Pontificia de Salamanca)\\
Universidad Tecnológica de Pereira\\
saulo.torres@utp.edu.co
	\end{itemize}
	\clearpage
	
	\section{Recursos disponibles}
	\begin{itemize}
		\item Computador portátil Dell Inspiron 1420\\
			Intel Centrino Duo, 2GB de memoria RAM
		\item Celular Nexus One, con sistema operativo Android\\
			Procesador ARM, 512 MB de memoria RAM, Cámara 5 MPixeles
		\item Base de datos de artículos publicados en IEEE explore
		\item Código fuente del OCR experimental de licencia libre Ocropus
	    \item Recursos bibliográficos de la biblioteca de la Universidad Tecnológica de Pereira y propios en el área de inteligencia artificial, reconocimiento óptico de caracteres y procesamiento digital de imágenes.
	\end{itemize}
	\clearpage
	
	\section{Cronograma}
	%el cronograma
	
\bibliographystyle{plain}
\bibliography{refs}
\nocite{*}
\end{document}
