%%%%%%%%%%%%%%%%%%%%%%%%%%%%%%%%%%%%%%%%%%%%%%%%%%%%%%%%%%%%%%%%%%%%%%%%%%%
%
% Plantilla para un artculo en LaTeX en espaol.
%
%%%%%%%%%%%%%%%%%%%%%%%%%%%%%%%%%%%%%%%%%%%%%%%%%%%%%%%%%%%%%%%%%%%%%%%%%%%

\documentclass[a4paper, 11pt, oneside]{article}


% idioma
\usepackage[utf8]{inputenc}
\usepackage[spanish]{babel}

%tablas
\usepackage{booktabs}

%rotar tablas
\usepackage{rotating}

%color tablas
\usepackage{colortbl}

%espaciado
\usepackage{setspace}
\onehalfspacing
\setlength{\parindent}{0pt}
\setlength{\parskip}{2.0ex plus0.5ex minus0.2ex}


%margenes según n. icontec
\usepackage{vmargin}
\setmarginsrb           { 4.0cm}  % left margin
                        { 4.0cm}  % top margcm
                        { 2.0cm}  % right margcm
                        { 3.0cm}  % bottom margcm
                        {  10pt}  % head height
                        {0.25cm}  % head sep
                        {   9pt}  % foot height
                        { 0.3cm}  % foot sep


% inserción url's notas de pie.
\usepackage{url}


% Paquetes de la AMS:
\usepackage{amsmath, amsthm, amsfonts}

% Teoremas
%--------------------------------------------------------------------------
\newtheorem{thm}{Teorema}[section]
\newtheorem{cor}[thm]{Corolario}
\newtheorem{lem}[thm]{Lema}
\newtheorem{prop}[thm]{Proposición}
\theoremstyle{definition}
\newtheorem{defn}[thm]{Definición}
\theoremstyle{remark}
\newtheorem{rem}[thm]{Observación}

% Atajos.
% Se pueden definir comandos nuevos para acortar cosas que se usan
% frecuentemente. Como ejemplo, aqu se definen la R y la Z dobles que
% suelen representar a los conjuntos de nmeros reales y enteros.
%--------------------------------------------------------------------------

\def\RR{\mathbb{R}}
\def\ZZ{\mathbb{Z}}

% De la misma forma se pueden definir comandos con argumentos. Por
% ejemplo, aqu definimos un comando para escribir el valor absoluto
% de algo ms fcilmente.
%--------------------------------------------------------------------------
\newcommand{\abs}[1]{\left\vert#1\right\vert}

% Operadores.
% Los operadores nuevos deben definirse como tales para que aparezcan
% correctamente. Como ejemplo definimos en jacobiano:
%--------------------------------------------------------------------------
\DeclareMathOperator{\Jac}{Jac}



\newcommand\portada{
\begin{titlepage}
		\begin{center}
			{\large \bf TITULO }
			\vfill
% 			{\large\bf PRESENTADO POR \par}
			{\large\bf AUTOR}
			\vfill
			{\large\bf MWISKA INSTITUTE OF TECHNOLOGY  \par}
			{\large\bf FACULTAD DE INGENIERÍA \par}
			{\large\bf CIENCIAS DE LA COMPUTACIÓN CON ENFASÍS EN FÍSICA Y CIBERNÉTICA \par}
			{\large\bf BOGOTÁ D.C.\par}
			{\large\bf AGOSTO 2008 \par}
		\end{center}
\end{titlepage}
}

\newcommand\contraportada{
	\begin{titlepage}
		\begin{center}
			{\large \bf TITULO } 
			\vfill
% 			{\large\bf PRESENTADO POR \par}
			{\large\bf AUTOR Cód 000000000}
			\vfill
			{\large\bf Anteproyecto de grado \par}
			\vfill
			{\large\bf Director: Título y NOMBRE DE SU DIRECTOR
\par}
			\vfill
			{\large\bf MWISKA INSTITUTE OF TECHNOLOGY \par}
			{\large\bf FACULTAD DE INGENIERÍA \par}
			{\large\bf CIENCIAS DE LA COMPUTACIÓN CON ENFASÍS EN FÍSICA Y CIBERNÉTICA \par}
			{\large\bf BOGOTÁ D.C.\par}
			{\large\bf AGOSTO 2008 \par}
		\end{center}
\end{titlepage}
}


%--------------------------------------------------------------------------
% \title{\LARGE \bf Plantilla para un artculo \LaTeX }
% % \vfill
% \author{El autor va aqu\\
%   \small Dept. Plantillas y Editores\\
%   \small E12345\\
%   \small Espaa
% }


\begin{document}
\portada
\contraportada


% \abstract{Esto es una plantilla simóíéññññple para un artculo en \LaTeX.}



\renewcommand\contentsname{\centering Tabla de Contenidos}
\tableofcontents
\clearpage

\begin{center}
 \section{Planteamiento del problema}
\end{center}

En ciencia o ingeniería los problemas no se formulan sino más bien se plantean. La formulación de
problemas puede caer en el ámbito de la fantasía.

Es fundamental identificar claramente la pregunta que se quiere responder o el problema
concreto a cuya solución o entendimiento se contribuirá con la ejecución del proyecto. Por lo tanto
se recomienda hacer una descripción precisa y completa de la naturaleza y magnitud del problema. De
igual manera se debe enunciar brevemente la solución al problema\footnote{Para estas
recomendaciones nos apoyamos en un documento antigüo, de vieja guardia de un \textit{lug} bogotano.
\textit{for-glud-aca}.}.

Todas las afirmaciones hechas en este documento, deben estar sustentadas por citas, referencias, o
resultados de experimentos reconocidos, es un documento de científico o de ingeniería, no es
literatura.

En lo posible no hable \textit{carreta} y no haga copias vulgares de contenido de Internet. Diga
mucho, escriba poco. Siga el ejemplo de John Forbes Nash con su famosa tesis de menos de 30
páginas\footnote{John Forbes Nash: matemático. \url{http://es.wikipedia.org/wiki/John_Forbes_Nash}.
Wikipedia, Wikimedia Foundation. Agosto 2008.}.
\clearpage


\begin{center}
 \section{Justificación}
\end{center}
Explica la razón por la cual el proyecto debe realizarse, se debe describir claramente porque la
necesidad del proyecto y debe hacer referencia a la solución del problema. Justifique de forma
precisa y clara, el impacto y aportes que el proyecto tendrá en el conocimiento actual del tema y
las soluciones para enfrentar el problema abordado.
\clearpage


\begin{center}
 \section{Marco Teórico}
\end{center}
Síntesis del contexto general (nacional y mundial) en el cual se ubica el tema de la
propuesta, estado actual del conocimiento del problema, brechas que existen y vacíos que se quiere
llenar con el proyecto. Incluye teorías, elementos conceptuales, manuales y trabajos previos que se
tomarán como referencias para la elaboración del trabajo.
\subsection{Antecedentes}
Que se ha hecho, dicho o escrito sobre el tema, situaciones o problemas en que se inscribe la
problemática del trabajo a desarrollar.
\begin{itemize}
  \item\textbf{algún item 0} -- item 0
  \item\textbf{algún item 1} -- item 1
\end{itemize}
\subsubsection{Referentes}
¿Qué trabajos se tomarán como referencia para fundamentar el sujeto de indagación?
\subsection{Marco conceptual}
Términos que serán necesarios precisar de forma operacional para no incurrir en confusiones o mal
interpretaciones\cite{Fogel05}.

Una expresión matemática: \begin{equation}\label{eq:area} S = \pi r^2\end{equation}.

Una cita textual: 
\begin{quote}
\textit{Por expresivos que sean, los símbolos no pueden ser las cosas que
representan.}\cite{Huxley54}
\end{quote}


\clearpage

\begin{center}
\section{Objetivos}
\end{center}
Los objetivos deben mostrar una relación clara y consistente con la descripción del problema y,
específicamente, con las preguntas y/o hipótesis que se quieren resolver. La formulación de
objetivos claros y viables constituye una base importante para juzgar el resto de la propuesta y
además facilita la estructuración de la metodología. Se recomienda formular un solo objetivo general
global, coherente con el problema planteado, y dos o más objetivos específicos que conducirán a
lograr el objetivo general y que son alcanzables con la metodología propuesta.


\subsection{Objetivo General}
Hace referencia a la explicación amplia y general de los resultados que se buscan obtener con la
ejecución del proyecto.

\subsection{Objetivos Específicos}
\begin{itemize}
\item En estrecha coherencia con el objetivo general...
\item ...escriba los propósitos concretos o resultados específicos que
\item ...se generarán del trabajo
\end{itemize}
\clearpage

\begin{center}
\section{Metodología}
\end{center}
Debe mostrar, en forma organizada, y precisa, cómo serán alcanzados cada uno de los objetivos
específicos propuestos. Deben detallarse, los procedimientos, técnicas, instrumentos, actividades,
etapas y demás estrategias metodológicas requeridas para el proyecto.  Tenga en cuenta que el
diseño metodológico es la base para planificar todas las actividades que demanda el proyecto y para
determinar los recursos humanos y financieros requeridos.

Un esquema útil puede ser:
\begin{enumerate}
  \item Un objetivo específico
  \begin{enumerate}
    \item se alcanza a través de unas etapas
    \begin{enumerate}
      \item que incluyen unos procedimientos
      \begin{itemize}
        \item que usan técnicas
        \item ... e instrumentos
      \end{itemize}
    \end{enumerate}
  \end{enumerate}
\end{enumerate}

Tuve la oportunidad de ver algunos trabajos radicados oficialmente, en esta parte dichos trabajos
vinculan las metodologías de desarrollo de software elegidas cuando son proyectos de desarrollo.
\clearpage


\begin{center}
\section{Alcances}
\end{center}
Aquí se realiza una descripción mas precisa de lo que pretende el proyecto y de los resultados
tangibles del mismo.  Estos deben estar de acuerdo con los objetivos planteados (pero no son una
reformulación de los objetivos) y ser coherentes con la metodología planteada, con las capacidades
del grupo de trabajo y con los medios de los cuales se disponga.
\clearpage

\begin{center}
\section{Limitaciones}
\end{center}
Se debe indicar lo que no se va a realizar en el proyecto y las situaciones que podrian llegar a
limitar el alcance de los objetivos.
\clearpage

\begin{center}
\section{Recursos}
\end{center}
Se debe indicar el recurso humano, fisico, tecnológico necesario y disponible para llevar a cabo el
proyecto.
\clearpage

\begin{sidewaystable}
\begin{center}
\section{Cronograma}
\end{center}


\begin{center}

\begin{tabular}{l p{4cm} l p{4cm} l p{2cm} l p{2cm} l p{2cm} c p{2cm} c p{2cm} }
\toprule
\rowcolor[gray]{0.9}Objetivo específico & Descripción & Dependencias & Recursos  & Producto &
Inicio & Final \\
\midrule
Objetivo 0 & descripción 0 & depende de & recursos & se obtiene & semana 0 & semana 1 \\
\midrule
Objetivo 1 & descripción 1 & depende de & recursos & se obtiene & semana 2 & semana 3 \\
\bottomrule
\end{tabular}

\end{center}
\end{sidewaystable}
\clearpage



% Bibliografa.
%-----------------------------------------------------------------

\begin{thebibliography}{99}


\bibitem{Fogel05} Fogel, Karl.\emph{Producing Open Source Software, How to Run a
Successsful Free Software Project} Karl Fogel. 2005.

\bibitem{Huxley54} Huxley Aldous.\emph{The Doors of Perception}. Edhasa. 1954.

\end{thebibliography}

\end{document}