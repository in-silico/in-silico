\documentclass{book}

\usepackage{verbatim}
\usepackage{mathtools}
\usepackage{amsmath}


\begin{document}

\chapter{Using BigNum library}
\label{chap:api}
The following is the structure of an arbitrary precision integer in the BigNum library.
\begin{verbatim}
/**
 * Structure that represents big integers, d[0] is the least significant
 * digit while d[size-1] is the most significant digit
 */
typedef struct {
    bn_word maxSize; /*Maximum number of digits that d can hold*/
    bn_word size; /*number of digits*/
    char sign; /*zero->negative, other->positive*/
    bn_word *d; /*value of each digits*/
} BigInt;
\end{verbatim}

This structure represents a common positional number system, so the Big Integer N is determined by the following expression:

\[N = \sum_{i=0}^{size-1}{d_i b^i} \quad\quad (0 \le d_i < b) \]

where $b$ represents the number system base, $d_i$  are the digits which represent the number N and $size$ is the amount of digits that represent N in base $b$.

In BigNum library the base $b$ is $2^N$, where $N$ is the bit length of a \verb+int+ data type in the enviroment used for running the library, normally this number is 16 for microcontrollers and 32 for general purpose CPU's and high level embedded systems.\\

Notice that the sign representation used in "BigNum" is the sign-magnitude representation, for this reason the big integer structure has a "sign" field.

It is important to mention that the $maxSize$ field establish a bound about the biggest and smallest number which can be represented, for example if a big integer $T$ has a $maxSize$ value of 50, this big integer can hold a number $k$ such that $-2^{50N}<k<2^{50N}$, pay attention to this for avoiding overflow errors in BigNum library.

\section{BigInteger creation}

\subsection{How to create an arbitrary precision }


\chapter{Intern Structure}

\section{Addition and subtraction}
For addition and subtraction the library uses the schoolbook algorithms. To support signed integer addition and subtraction, the library has methods implemented for unsigned addition and subtraction. The function for unsigned addition is:
    \begin{verbatim}
    bnUAddInt(BigInt *res, BigInt *a, BigInt *b)
    \end{verbatim}
The function receives two integers \verb+a+ and \verb+b+ and adds them ignoring the signs in \verb+res+. Let $a_i$ and $b_i$ be the digit $i$-th of the first and second operands respectively, and $c_i$ be the $i$-th digit of the result, then the internal behavior of this function is given by:
\[c_i = a_i + b_i + k \]
where $k$ is the carry digit of the operation on the $(i-1)$-th digit. The function for unsigned subtraction is:
    \begin{verbatim}
    bnUSubInt(BigInt *res, BigInt *a, BigInt *b)
    \end{verbatim}
The function receives two integers \verb+a+ and \verb+b+ and subtracts them in \verb+res+ as with natural numbers, the magnitude of \verb+a+ must be greater or equal to \verb+b+ to avoid a less than zero result. Let $a_i$ and $b_i$ be the digit $i$-th of the first and second operands respectively, and $c_i$ be the $i$-th digit of the result, then the internal behavior of this function is given by:
\[c_i = a_i - b_i - k \]
where $k$ is the borrow digit of the operation on the $(i-1)$-th digit. \\

With the operations for addition and subtraction on natural numbers implemented, is easy to implement a signed addition with the following algorithm:
\begin{itemize}
\item If the signs of both operands are equal, then use unsigned addition and put the sign of the operands to the results
\item If the signs are different, then use unsigned subtraction to subtract the operand of lowest magnitude from the operand of highest magnitude and keep the sign of the operand of highest magnitude.
\end{itemize}

With the signed addition implemented, the signed subtraction was implemented by changing the sign of the second operand and calling the signed addition. The signature of the functions for signed addition and subtraction was explained in the chapter \ref{chap:api}.

\section{Multiplication}

\section{Division}
 
\end{document}
