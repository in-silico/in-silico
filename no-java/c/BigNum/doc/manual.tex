\documentclass{book}

\usepackage{verbatim}
\usepackage{mathtools}
\usepackage{amsmath}


\begin{document}

\chapter{Using BigNum library}
The following is the structure of an arbitrary precision integer in the BigNum library.
\begin{verbatim}
/**
 * Structure that represents big integers, d[0] is the least significant
 * digit while d[size-1] is the most significant digit
 */
typedef struct {
    bn_word maxSize; /*Maximum number of digits that d can hold*/
    bn_word size; /*number of digits*/
    char sign; /*zero->negative, other->positive*/
    bn_word *d; /*value of each digits*/
} BigInt;
\end{verbatim}

This structure represents a common positional number system, so the Big Integer N is determined by the following expression:

\[N = \sum_{i=0}^{size-1}{d_i b^i} \quad\quad (0 \le d_i < b) \]

where $b$ represents the number system base, $d_i$  are the digits which represent the number N and $size$ is the amount of digits that represent N in base $b$.

In BigNum library the base $b$ is $2^N$, where $N$ is the bit length of a \verb+int+ data type in the enviroment used for running the library, normally this number is 16 for microcontrollers and 32 for general purpose CPU's and high level embedded systems.\\

Notice that the sign representation used in "BigNum" is the sign-magnitude representation, for this reason the big integer structure has a "sign" field.

It is important to mention that the $maxSize$ field establish a bound about the biggest and smallest number which can be represented, for example if a big integer $T$ has a $maxSize$ value of 50, this big integer can hold a number $k$ such that $-2^{50N}<k<2^{50N}$, pay attention to this for avoiding overflow errors in BigNum library.

\section{BigInteger creation}

\subsection{How to create an arbitrary precision }


\chapter{Intern Structure}
 
\end{document}
